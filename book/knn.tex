\chapter{Geometry and Nearest Neighbors} \label{sec:knn}

\chapterquote{Our brains have evolved to get us out of the rain, find where the berries are, and keep us from getting killed. Our brains did not evolve to help us grasp really large numbers or to look at things in a hundred thousand dimensions.}{Ronald~Graham}


\begin{learningobjectives}
\item Describe a data set as points in a high dimensional space.
\item Explain the curse of dimensionality.
\item Compute distances between points in high dimensional space.
\item Implement a $K$-nearest neighbor model of learning.
\item Draw decision boundaries.
\item Implement the $K$-means algorithm for clustering.
\end{learningobjectives}

\dependencies{\chref{sec:dt}}

\newthought{You can think of prediction tasks} as mapping inputs
(course reviews) to outputs (course ratings).  As you learned in the
previous chapter, decomposing an input into a collection of features
(e.g., words that occur in the review) forms a useful abstraction for
learning.  Therefore, inputs are nothing more than lists of feature
values.  This suggests a \concept{geometric view} of data, where we
have one dimension for every feature.  In this view, examples are
points in a high-dimensional space.

Once we think of a data set as a collection of points in high
dimensional space, we can start performing geometric operations on
this data.  For instance, suppose you need to predict whether Alice
will like Algorithms.  Perhaps we can try to find another student who
is most ``similar'' to Alice, in terms of favorite courses.  Say this
student is Jeremy.  If Jeremy liked Algorithms, then we might guess
that Alice will as well.  This is an example of a \concept{nearest
  neighbor} model of learning.  By inspecting this model, we'll see a
completely different set of answers to the key learning questions we
discovered in \chref{sec:dt}.

\section{From Data to Feature Vectors}

An example is just a collection of feature values about that example,
for instance the data in Table~\ref{tab:data:course} from the Appendix.
To a person, these features have meaning.  One feature might count how
many times the reviewer wrote ``excellent'' in a course review.
Another might count the number of exclamation points.  A third might
tell us if any text is underlined in the review.

To a machine, the \concept{features} themselves have no meaning.  Only
the \concept{feature values}, and how they vary across examples, mean
something to the machine.  From this perspective, you can think about
an example as being represented by a \concept{feature vector}
consisting of one ``dimension'' for each feature, where each dimenion
is simply some real value.

Consider a review that said ``excellent'' three times, had one
exclamation point and no underlined text.  This could be represented
by the feature vector $\langle 3, 1, 0 \rangle$.  An almost identical
review that happened to have underlined text would have the feature
vector $\langle 3,1,1\rangle$.

Note, here, that we have imposed the convention that for
\concept{binary features} (yes/no features), the corresponding feature
values are $0$ and $1$, respectively.  This was an arbitrary choice.
We could have made them $0.92$ and $-16.1$ if we wanted.  But $0/1$ is
convenient and helps us interpret the feature values.  When we discuss
practical issues in Chapter~\ref{sec:prac}, you will see other reasons
why $0/1$ is a good choice.

\Figure{knn_projections}{A figure showing projections of data in two
  dimension in three ways -- see text.  Top: horizontal axis
  corresponds to the first feature (easy) and the vertical axis
  corresponds to the second feature (AI?); Middle: horizontal is
  second feature and vertical is third (systems?); Bottom: horizontal
  is first and vertical is third. Truly, the data points would like
  exactly on $(0,0)$ or $(1,0)$, etc., but they have been purturbed
  slightly to show duplicates.}

Figure~\ref{fig:knn_projections} shows the data from
Table~\ref{tab:data:course} in three views.  These three views are
constructed by considering two features at a time in different pairs.
In all cases, the plusses denote positive examples and the minuses
denote negative examples.  In some cases, the points fall on top of
each other, which is why you cannot see 20 unique points in all
figures.

\thinkaboutit{Match the example ids from Table~\ref{tab:data:course}
  with the points in Figure~\ref{fig:knn_projections}.}

The mapping from feature values to vectors is straighforward in the
case of real valued features (trivial) and binary features (mapped to
zero or one).  It is less clear what to do with \concept{categorical
  features}. For example, if our goal is to identify whether an object
in an image is a tomato, blueberry, cucumber or cockroach, we might
want to know its color: is it \categ{Red}, \categ{Blue}, \categ{Green}
or \categ{Black}?

One option would be to map \categ{Red} to a value of $0$, \categ{Blue}
to a value of $1$, \categ{Green} to a value of $2$ and \categ{Black}
to a value of $3$.  The problem with this mapping is that it turns an
unordered set (the set of colors) into an ordered set (the set
$\{0,1,2,3\}$).  In itself, this is not necessarily a bad thing.  But
when we go to \emph{use} these features, we will measure examples
based on their distances to each other.  By doing this mapping, we are
essentially saying that \categ{Red} and \categ{Blue} are more similar
(distance of $1$) than \categ{Red} and \categ{Black} (distance of
$3$).  This is probably not what we want to say!

A solution is to turn a categorical feature that can take four
different values (say: \categ{Red}, \categ{Blue}, \categ{Green} and
\categ{Black}) into four binary features (say: IsItRed?, IsItBlue?,
IsItGreen? and IsItBlack?).  In general, if we start from a
categorical feature that takes $V$ values, we can map it to $V$-many
binary indicator features.

\thinkaboutit{The computer scientist in you might be saying: actually
  we could map it to $\log_2 V$-many binary features!  Is this a good
  idea or not?}

With that, you should be able to take a data set and map each example
to a feature vector through the following mapping:

\begin{itemize}
\item Real-valued features get copied directly.
\item Binary features become $0$ (for false) or $1$ (for true).
\item Categorical features with $V$ possible values get mapped to
  $V$-many binary indicator features.
\end{itemize}

After this mapping, you can think of a single example as a
\concept{vector} in a high-dimensional \concept{feature space}.  If
you have $D$-many features (after expanding categorical features),
then this \concept{feature vector} will have $D$-many components.  We
will denote feature vectors as $\vec x = \langle x_1, x_2, \dots, x_D
\rangle$, so that $x_d$ denotes the value of the $d$th feature of
$\vec x$.  Since these are vectors with real-valued components in
$D$-dimensions, we say that they belong to the space $\R^D$.

For $D=2$, our feature vectors are just points in the plane, like in
Figure~\ref{fig:knn_projections}.  For $D=3$ this is three dimensional
space.  For $D>3$ it becomes quite hard to visualize.  (You should
resist the temptation to think of $D=4$ as ``time'' -- this will just
make things confusing.)  Unfortunately, for the sorts of problems you
will encounter in machine learning, $D \approx 20$ is considered ``low
dimensional,'' $D \approx 1000$ is ``medium dimensional'' and $D
\approx 100000$ is ``high dimensional.''

\thinkaboutit{Can you think of problems (perhaps ones already
  mentioned in this book!) that are low dimensional?  That are medium
  dimensional?  That are high dimensional?}

\section[K-Nearest Neighbors]{$K$-Nearest Neighbors}

The biggest advantage to thinking of examples as vectors in a high
dimensional space is that it allows us to apply geometric concepts to
machine learning.  For instance, one of the most basic things that one
can do in a vector space is compute \koncept{distances}{distance}.  In
two-dimensional space, the distance between $\langle 2,3\rangle$ and
$\langle 6,1\rangle$ is given by $\sqrt{(2-6)^2 + (3-1)^2} = \sqrt{18}
\approx 4.24$.  In general, in $D$-dimensional space, the
\concept{Euclidean distance} between vectors $\vec a$ and $\vec b$ is
given by Eq~\eqref{eq:euclidean} (see Figure~\ref{fig:knn_euclidean}
for geometric intuition in three dimensions):


\begin{equation} \label{eq:euclidean}
d(\vec a, \vec b) = \left[ \sum_{d=1}^D (a_d - b_d)^2 \right]^{\frac 1 2}
\end{equation}

\Figure{knn_euclidean}{A figure showing Euclidean distance in
  three dimensions. The length of the green segments are $0.6$, $0.6$ and $0.3$ respectively, in the x-, y-, and z-axes. The total distance between the red dot and the orange dot is therefore $\sqrt{0.6^2 + 0.6^2 + 0.3^2} = 0.9$.}

\thinkaboutit{Verify that $d$ from Eq~\eqref{eq:euclidean} gives the
  same result ($4.24$) for the previous computation.}

\Figure{knn_classifyit}{A figure showing an easy NN classification
  problem where the test point is a ? and should be negative.}

Now that you have access to distances between examples, you can start
thinking about what it means to learn again.  Consider
Figure~\ref{fig:knn_classifyit}.  We have a collection of training
data consisting of positive examples and negative examples.  There is
a test point marked by a question mark.  Your job is to guess the
correct label for that point.

Most likely, you decided that the label of this test point is
positive.  One reason why you might have thought that is that you
believe that the label for an example should be similar to the label
of nearby points.  This is an example of a new form of
\concept{inductive bias}.

The \concept{nearest neighbor} classifier is build upon this insight.
In comparison to decision trees, the algorithm is ridiculously
simple.  At training time, we simply store the entire training set.
At test time, we get a test example $\hat\vx$.  To predict its label,
we find the training example $\vx$ that is most similar to $\hat\vx$.
In particular, we find the training example $\vx$ that
\emph{minimizes} $d(\vx,\hat\vx)$.  Since $\vx$ is a training example,
it has a corresponding label, $y$.  We predict that the label of
$\hat\vx$ is also $y$.

Despite its simplicity, this nearest neighbor classifier is incredibly
effective.  (Some might say \emph{frustratingly} effective.)  However,
it is particularly prone to overfitting label noise.  Consider the
data in Figure~\ref{fig:knn_classifyitbad}.  You would probably want
to label the test point positive.  Unfortunately, it's nearest
neighbor happens to be negative.  Since the nearest neighbor algorithm
only looks at the \emph{single} nearest neighbor, it cannot consider
the ``preponderance of evidence'' that this point should probably
actually be a positive example.  It will make an unnecessary error.

\Figure{knn_classifyitbad}{A figure showing an easy NN
  classification problem where the test point is a ? and should be
  positive, but its NN is actually a negative point that's noisy.}


A solution to this problem is to consider more than just the single
nearest neighbor when making a classification decision.  We can
consider the \koncept{$K$-nearest neighbors}{K-nearest neighbors} and
let them \concept{vote} on the correct class for this test point.  If
you consider the $3$-nearest neighbors of the test point in
Figure~\ref{fig:knn_classifyitbad}, you will see that two of them are
positive and one is negative.  Through voting, positive would win.

\thinkaboutit{Why is it a good idea to use an odd number for $K$?}

\newalgorithm%
  {knn:knn}%
  {\FUN{KNN-Predict}(\VAR{$\mat D$}, \VAR{K}, \VAR{$\hat\vx$})}
  {
\SETST{S}{\emptylist}
\FOR{\VAR{$n$} $=\CON{1}$ \TO \VAR{$N$}}
\SETST{S}{\VAR{S} \pushlist{$\langle$d(\VAR{$\vx_n$}, \VAR{$\hat\vx$}), \VAR{n}$\rangle$}} \COMMENT{store distance to training example $n$}
\ENDFOR
\SETST{S}{\FUN{sort}(\VAR{S})} \COMMENT{put lowest-distance objects first}
\SETST{$\hat y$}{\CON{0}}
\FOR{\VAR{k} $=\CON{1}$ \TO \VAR{K}}
\SETST{$\langle$dist,n$\rangle$}{\VAR{S$_k$}} \COMMENT{$n$ this is the $k$th closest data point}
\SETST{$\hat y$}{\VAR{$\hat y$} $+$ \VAR{$y_n$}} \COMMENT{vote
  according to the label for the $n$th training point}
\ENDFOR
\RETURN \FUN{sign}(\VAR{$\hat y$}) \COMMENT{return $+1$ if $\hat y>0$
  and $-1$ if $\hat y<0$}
}

The full algorithm for $K$-nearest neighbor classification is given in
Algorithm~\ref{alg:knn:knn}.  Note that there actually is no
``training'' phase for $K$-nearest neighbors.  In this algorithm we
have introduced five new conventions:

\begin{enumerate}
\item The training data is denoted by \VAR{$\mat D$}.
\item We assume that there are $N$-many training examples.
\item These examples are pairs $(\vx_1,y_1), (\vx_2,y_2), \dots,
  (\vx_N,y_N)$.\\(Warning: do not confuse $\vx_n$, the $n$th training
  example, with $x_d$, the $d$th feature for example $\vx$.)
\item We use \emptylist to denote an empty list and \pushlist{$\cdot$}
  to append $\cdot$ to that list.
\item Our prediction on $\hat\vx$ is called $\hat y$.
\end{enumerate}

The first step in this algorithm is to compute distances from the test
point to all training points (lines 2-4).  The data points are then
sorted according to distance.  We then apply a clever trick of
\emph{summing} the class labels for each of the $K$ nearest neighbors
(lines 6-10) and using the \FUN{sign} of this sum as our prediction.

\thinkaboutit{Why is the sign of the sum computed in lines 2-4 the
  same as the majority vote of the associated training examples?}

The big question, of course, is how to choose $K$.  As we've seen,
with $K=1$, we run the risk of overfitting.  On the other hand, if $K$
is large (for instance, $K=N$), then \FUN{KNN-Predict} will always
predict the majority class.  Clearly that is underfitting.  So, $K$ is
a hyperparameter of the KNN algorithm that allows us to trade-off
between overfitting (small value of $K$) and underfitting (large value
of $K$).

\thinkaboutit{Why can't you simply pick the value of $K$ that does
  best on the training data?  In other words, why do we have to treat
  it like a hyperparameter rather than just a parameter.}

One aspect of \concept{inductive bias} that we've seen for KNN is that
it assumes that nearby points should have the same label.  Another
aspect, which is quite different from decision trees, is that all
features are equally important!  Recall that for decision trees, the
key question was \emph{which features are most useful for
  classification?}  The whole learning algorithm for a decision tree
hinged on finding a small set of good features.  This is all thrown
away in KNN classifiers: every feature is used, and they are all used
the same amount.  This means that if you have data with only a few
relevant features and lots of irrelevant features, KNN is likely to do
poorly.

\MoveNextFigure{-18cm}
\Figure{knn:ski}{A figure of a ski and a snowboard.}

\MoveNextFigure{-10cm}
\Figure{knn:skidata}{Classification data for ski vs snowboard in
  2d}

A related issue with KNN is \concept{feature scale}.  Suppose that we
are trying to classify whether some object is a ski or a snowboard
(see Figure~\ref{fig:knn:ski}).  We are given two features about this
data: the width and height.  As is standard in skiing, width is
measured in millimeters and height is measured in centimeters.  Since
there are only two features, we can actually plot the entire training
set; see Figure~\ref{fig:knn:skidata} where ski is the positive class.
Based on this data, you might guess that a KNN classifier would do
well.

Suppose, however, that our measurement of the width was computed in
millimeters (instead of centimeters).  This yields the data shown in
Figure~\ref{fig:knn:skidatabad}.  Since the width values are now tiny,
in comparison to the height values, a KNN classifier will effectively
\emph{ignore} the width values and classify almost purely based on
height.  The predicted class for the displayed test point had changed
because of this feature scaling.

\MoveNextFigure{-5cm}
\Figure{knn:skidatabad}{Classification data for ski vs snowboard in
  2d, with width rescaled to mm.}


We will discuss feature scaling more in Chapter~\ref{sec:prac}.  For
now, it is just important to keep in mind that KNN does not have the
power to decide which features are important.

\begin{mathreview}{Vector Arithmetic and Vector Norms}
  A (real-valued) \textbf{vector} is just an array of real values, for
  instance $\vx = \langle 1, 2.5, -6 \rangle$ is a three-dimensional
  vector. In general, if $\vx = \langle x_1, x_2, \dots, x_D \rangle$,
  then $x_d$ is it's $d$th component. So $x_3 = -6$ in the previous
  example.

~

  \textbf{Vector sums} are computed pointwise, and are only defined when
  dimensions match, so $\langle 1, 2.5, -6 \rangle + \langle 2, -2.5,
  3 \rangle = \langle 3, 0, -3 \rangle$. In general, if $\vec c = \vec
  a + \vec b$ then $c_d = a_d + b_d$ for all $d$. Vector addition can
  be viewed geometrically as taking a vector $\vec a$, then tacking on
  $\vec b$ to the end of it; the new end point is exactly $\vec c$.

~

  Vectors can be \textbf{scaled} by real values; for instance $2
  \langle 1, 2.5, -6 \rangle = \langle 2, 5, -12 \rangle$; this is
  called scalar multiplication. In general, $a \vx = \langle a x_1, a
  x_2, \dots, a x_D \rangle$.

~

  The \textbf{norm} of a vector $\vx$, written $\norm{\vx}$ is its
  length. Unless otherwise specified, this is its \emph{Euclidean}
  length, namely: $\norm{\vx} = \sqrt{\sum_d x_d^2}$.
\end{mathreview}

\section{Decision Boundaries}

The standard way that we've been thinking about learning algorithms up
to now is in the \emph{query model}.  Based on training data, you
learn something.  I then give you a query example and you have to
guess it's label.

\Figure{knn:db}{decision boundary for 1nn.}

An alternative, less passive, way to think about a learned model is to
ask: what sort of test examples will it classify as positive, and what
sort will it classify as negative.  In Figure~\ref{fig:knn:db}, we have a
set of training data.  The background of the image is colored blue in
regions that \emph{would} be classified as positive (if a query were
issued there) and colored red in regions that \emph{would} be
classified as negative.  This coloring is based on a $1$-nearest
neighbor classifier.

In Figure~\ref{fig:knn:db}, there is a solid line separating the positive
regions from the negative regions.  This line is called the
\concept{decision boundary} for this classifier.  It is the line with
positive land on one side and negative land on the other side.

\Figure{knn:db3}{decision boundary for knn with k=3.}

Decision boundaries are useful ways to visualize the
\concept{complexity} of a learned model.  Intuitively, a learned model
with a decision boundary that is really jagged (like the coastline of
Norway) is really complex and prone to overfitting.  A learned model
with a decision boundary that is really simple (like the bounary
between Arizona and Utah) is potentially underfit.
%In
%Figure~\ref{fig:knn:dbmany}, you can see the decision boundaries for KNN
%models with $K \in \{1, 3, 5, 7\}$.  As you can see, the boundaries
%become simpler and simpler as $K$ gets bigger.

Now that you know about decision boundaries, it is natural to ask:
what do decision boundaries for decision trees look like?  In order to
answer this question, we have to be a bit more formal about how to
build a decision tree on real-valued features.  (Remember that the
algorithm you learned in the previous chapter implicitly assumed
\emph{binary} feature values.)  The idea is to allow the decision tree
to ask questions of the form: ``is the value of feature $5$ greater
than $0.2$?''  That is, for real-valued features, the decision tree
nodes are parameterized by a feature and a threshold for that
feature.  An example decision tree for classifying skis versus
snowboards is shown in Figure~\ref{fig:knn:dtski}.

\Figure{knn:dtski}{decision tree for ski vs. snowboard}

\Figure{knn:dtdb}{decision boundary for dt in previous figure}

Now that a decision tree can handle feature vectors, we can talk about
decision boundaries.  By example, the decision boundary for the
decision tree in Figure~\ref{fig:knn:dtski} is shown in
Figure~\ref{fig:knn:dtdb}.  In the figure, space is first split in half
according to the first query along one axis.  Then, depending on which
half of the space you look at, it is either split again along the
other axis, or simply classified.

%\TODOFigure{knn:dtdeepdb}{decision boundary for deep dt on ski data}

Figure~\ref{fig:knn:dtdb} is a good visualization of decision boundaries
for decision trees in general.  Their decision boundaries are
axis-aligned cuts.  The cuts must be axis-aligned because nodes can
only query on a single feature at a time.  In this case, since the
decision tree was so shallow, the decision boundary was relatively
simple.

\thinkaboutit{What sort of data might yield a very simple decision
  boundary with a decision tree and very complex decision boundary
  with 1-nearest neighbor?  What about the other way around?}

\section[K-Means Clustering]{$K$-Means Clustering}
\label{sec:knn:kmeans}

Up through this point, you have learned all about supervised learning
(in particular, binary classification).  As another example of the use
of geometric intuitions and data, we are going to temporarily consider
an \concept{unsupervised learning} problem.  In unsupervised learning,
our data consists \emph{only} of examples $\vx_n$ and does \emph{not}
contain corresponding labels.  Your job is to make sense of this data,
even though no one has provided you with correct labels.  The
particular notion of ``making sense of'' that we will talk about now
is the \concept{clustering} task.

\Figure{knn:clustering}{simple clustering data... clusters in UL,
  UR and BC.}

Consider the data shown in Figure~\ref{fig:knn:clustering}.  Since
this is unsupervised learning and we do not have access to labels, the
data points are simply drawn as black dots.  Your job is to split this
data set into three clusters.  That is, you should label each data
point as \categ{A}, \categ{B} or \categ{C} in whatever way you want.

For this data set, it's pretty clear what you should do.  You probably
labeled the upper-left set of points \categ{A}, the upper-right set of
points \categ{B} and the bottom set of points \categ{C}.  Or perhaps
you permuted these labels.  But chances are your clusters were the
same as mine.

The $K$-means clustering algorithm is a particularly simple and
effective approach to producing clusters on data like you see in
Figure~\ref{fig:knn:clustering}.  The idea is to represent each
cluster by it's cluster center.  Given cluster centers, we can simply
assign each point to its nearest center.  Similarly, if we know the
assignment of points to clusters, we can compute the centers.  This
introduces a chicken-and-egg problem.  If we knew the clusters, we
could compute the centers.  If we knew the centers, we could compute
the clusters.  But we don't know either.

\Figure{knn:kmeans}{first few iterations of k-means running on
  previous data set}

The general computer science answer to chicken-and-egg problems is
\concept{iteration}.  We will start with a guess of the cluster
centers.  Based on that guess, we will assign each data point to its
closest center.  Given these new assignments, we can recompute the
cluster centers.  We repeat this process until clusters stop moving.
The first few iterations of the $K$-means algorithm are shown in
Figure~\ref{fig:knn:kmeans}.  In this example, the clusters converge
very quickly.

\newalgorithm%
  {knn:kmeans}%
  {\FUN{K-Means}(\VAR{$\mat D$}, \VAR{K})}
  {
\FOR{$\VAR{k} = \CON{1}$ \TO $\VAR{K}$}
\SETST{$\vec\mu_k$}{some random location} \COMMENT{randomly initialize
  center for $k$th cluster}
\ENDFOR
\REPEAT
\FOR{$\VAR{n} = \CON{1}$ \TO $\VAR{N}$}
\SETST{$z_n$}{$\argmin_k \norm{\vec\mu_k - \vx_n}$}
  \COMMENT{assign example $n$ to closest center}
\ENDFOR
\FOR{$\VAR{k} = \CON{1}$ \TO $\VAR{K}$}
\SETST{$\mat X_k$}{$\{$ \VAR{$\vx_n$} : \VAR{$z_n$} = \VAR{$k$} $\}$}
  \COMMENT{points assigned to cluster $k$}
\SETST{$\vec\mu_k$}{\FUN{mean}(\VAR{$\mat X_k$})}
  \COMMENT{re-estimate center of cluster $k$}
\ENDFOR
\UNTIL{\VAR{$\vec\mu$}s stop changing}
\RETURN \VAR{$\vec z$} \COMMENT{return cluster assignments}
}

Algorithm~\ref{alg:knn:kmeans} spells out the $K$-means clustering
algorithm in detail.  The cluster centers are initialized randomly.
In line 6, data point $\vx_n$ is compared against each cluster center
$\vec\mu_k$.  It is assigned to cluster $k$ if $k$ is the center with
the smallest distance.  (That is the ``$\argmin$'' step.)  The
variable \VAR{$z_n$} stores the assignment (a value from $1$ to $K$)
of example $n$.  In lines 8-12, the cluster centers are re-computed.
First, $\mat X_k$ stores all examples that have been assigned to
cluster $k$.  The center of cluster $k$, \VAR{$\vec\mu_k$} is then
computed as the mean of the points assigned to it.  This process
repeats until the centers converge.

An obvious question about this algorithm is: does it converge?  A
second question is: how long does it take to converge.  The first
question is actually easy to answer.  Yes, it does.  And in practice,
it usually converges quite quickly (usually fewer than $20$
iterations).  In Chapter~\ref{sec:unsup}, we will actually
\emph{prove} that it converges.  The question of how long it takes to
converge is actually a really interesting question.  Even though the
$K$-means algorithm dates back to the mid 1950s, the best known
convergence rates were \emph{terrible} for a long time.  Here,
terrible means exponential in the number of data points.  This was a
sad situation because empirically we knew that it converged very
quickly.  New algorithm analysis techniques called ``smoothed
analysis'' were invented in 2001 and have been used to show very fast
convergence for $K$-means (among other algorithms).  These techniques
are well beyond the scope of this book (and this author!) but suffice
it to say that $K$-means is fast in practice and is provably fast in
theory.

%\TODOFigure{knn:kmeansbad}{first few iterations of k-means running on
%  previous data set with BAD initialization}

It is important to note that although $K$-means is guaranteed to
converge and guaranteed to converge quickly, it is \emph{not}
guaranteed to converge to the ``right answer.''  The key problem with
unsupervised learning is that we have no way of knowing what the
``right answer'' is.  Convergence to a bad solution is usually due to
poor initialization.
%For example, poor initialization in the data set
%from before yields convergence like that seen in
%Figure~\ref{fig:knn:kmeansbad}.  As you can see, the algorithm
%\emph{has} converged.  It has just converged to something less than
%satisfactory.

\thinkaboutit{What is the difference between unsupervised and
  supervised learning that means that we know what the ``right
  answer'' is for supervised learning but not for unsupervised
  learning?}

\section{Warning: High Dimensions are Scary}

Visualizing one hundred dimensional space is incredibly difficult for
humans.  After huge amounts of training, some people have reported
that they can visualize four dimensional space in their heads.  But
beyond that seems impossible.\sidenote{If you want to try to get an
  intuitive sense of what four dimensions looks like, I highly
  recommend the short 1884 book \emph{Flatland: A Romance of Many
    Dimensions} by Edwin Abbott Abbott.  You can even read it online
  at \url{gutenberg.org/ebooks/201}.}

In addition to being hard to visualize, there are at least two
additional problems in high dimensions, both refered to as
\concept{the curse of dimensionality}.  One is computational, the
other is mathematical.

\Figure{knn:grid:2d}{2d knn with an overlaid grid, cell with test
  point highlighted}

From a computational perspective, consider the following problem.
For $K$-nearest neighbors, the speed of prediction is slow for a very
large data set.  At the very least you have to look at every training
example every time you want to make a prediction.  To speed things up
you might want to create an \emph{indexing} data structure.  You can
break the plane up into a grid like that shown in
Figure~\ref{fig:knn:grid:2d}.  Now, when the test point comes in, you can
quickly identify the grid cell in which it lies.  Now, instead of
considering \emph{all} training points, you can limit yourself to
training points \emph{in that grid cell} (and perhaps the neighboring
cells).  This can potentially lead to huge computational savings.

In two dimensions, this procedure is effective.  If we want to break
space up into a grid whose cells are $0.2 \times 0.2$, we can clearly
do this with $25$ grid cells in two dimensions (assuming the range of
the features is $0$ to $1$ for simplicity).  In three dimensions,
we'll need $125 = 5 \times 5 \times 5$ grid cells.  In four
dimensions, we'll need $625$.  By the time we get to ``low
dimensional'' data in $20$ dimensions, we'll need $95,367,431,640,625$
grid cells (that's $95$ trillion, which is about $6$ to $7$ times the
US national debt as of January 2011).  So if you're in $20$
dimensions, this gridding technique will only be useful if you have at
least $95$ trillion training examples.

For ``medium dimensional'' data (approximately $1000$) dimesions, the
number of grid cells is a $9$ followed by $698$ numbers before the
decimal point.  For comparison, the number of atoms in the universe is
approximately $1$ followed by $80$ zeros.  So even if each atom
yielded a googul training examples, we'd still have far fewer
examples than grid cells.  For ``high dimensional'' data
(approximately $100000$) dimensions, we have a $1$ followed by just
under $70,000$ zeros.  Far too big a number to even really comprehend.

Suffice it to say that for even moderately high dimensions, the amount
of computation involved in these problems is enormous.

\thinkaboutit{How does the above analysis relate to the number of data
  points you would need to fill out a full decision tree with $D$-many
  features?  What does this say about the importance of shallow
  trees?}

In addition to the computational difficulties of working in high
dimensions, there are a large number of strange mathematical
occurances there.  In particular, many of your intuitions that you've
built up from working in two and three dimensions just do not carry
over to high dimensions.  We will consider two effects, but there are
countless others.  The first is that high dimensional spheres look
more like porcupines than like balls.\sidenote{This result was
  related to me by Mark Reid, who heard about it from Marcus Hutter.}
The second is that distances between points in high dimensions are all
approximately the same.

\Figure{knn:cursetwo}{2d spheres in spheres}

Let's start in two dimensions as in Figure~\ref{fig:knn:cursetwo}.
We'll start with four green spheres, each of radius one and each
touching exactly two other green spheres.  (Remember that in two
dimensions a ``sphere'' is just a ``circle.'')  We'll place a red
sphere in the middle so that it touches all four green spheres.  We
can easily compute the radius of this small sphere.  The pythagorean
theorem says that $1^2 + 1^2 = (1+r)^2$, so solving for $r$ we get $r
= \sqrt 2 - 1 \approx 0.41$.  Thus, by calculation, the blue sphere
lies entirely within the cube (cube = square) that contains the grey
spheres.  (Yes, this is also obvious from the picture, but perhaps you
can see where this is going.)

\Figure{knn:cursethree}{3d spheres in spheres}

Now we can do the same experiment in three dimensions, as shown in
Figure~\ref{fig:knn:cursethree}.  Again, we can use the pythagorean
theorem to compute the radius of the blue sphere.  Now, we get $1^2 +
1^2 + 1^2 = (1+r)^2$, so $r = \sqrt3 - 1 \approx 0.73$.  This is still
entirely enclosed in the cube of width four that holds all eight grey
spheres.

At this point it becomes difficult to produce figures, so you'll have
to apply your imagination.  In four dimensions, we would have $16$
green spheres (called \concept{hyperspheres}), each of radius one.
They would still be inside a cube (called a \concept{hypercube}) of
width four.  The blue hypersphere would have radius $r = \sqrt4 - 1 =
1$.  Continuing to five dimensions, the blue hypersphere embedded in
$256$ green hyperspheres would have radius $r = \sqrt5-1 \approx 1.23$
and so on.

In general, in $D$-dimensional space, there will be $2^D$ green
hyperspheres of radius one.  Each green hypersphere will touch exactly
$n$-many other hyperspheres.  The blue hyperspheres in the middle will
touch them all and will have radius $r = \sqrt D - 1$.

Think about this for a moment.  As the number of dimensions grows, the
radius of the blue hypersphere \emph{grows without bound!}.  For
example, in $9$-dimensions the radius of the blue hypersphere is now
$\sqrt9-1 = 2$.  But with a radius of two, the blue hypersphere is now
``squeezing'' between the green hypersphere and \emph{touching} the
edges of the hypercube.  In $10$ dimensional space, the radius is
approximately $2.16$ and it pokes outside the cube.

%\Figure{knn:porcupine}{porcupine versus ball}

%This is why we say that high dimensional spheres look like porcupines
%and not balls (see Figure~\ref{fig:knn:porcupine}).  The moral of this
%story from a \emph{machine learning} perspective is that intuitions
%you have about space might not carry over to high dimensions.  For
%example, what you think looks like a ``round'' cluster in two or three
%dimensions, might not look so ``round'' in high dimensions.

%\Figure{knn:uniform}{100 uniform random points in 1, 2 and 3
%  dimensions}

The second strange fact we will consider has to do with the distances
between points in high dimensions.  We start by considering random
points in one dimension.  That is, we generate a fake data set
consisting of $100$ random points between zero and one.  We can do the
same in two dimensions and in three dimensions.  See
Figure~\ref{fig:knn:uniform} for data distributed uniformly on the
\concept{unit hypercube} in different dimensions.

Now, pick two of these points at random and compute the distance
between them.  Repeat this process for all pairs of points and average
the results.  For the data shown in Figure~\ref{fig:knn:uniform}, the
average distance between points in one dimension is about $0.346$; in two
dimensions is about $0.518$; and in three dimensions is $0.615$. The fact that these \emph{increase} as the dimension increases is not surprising. The furthest two points can be in a 1-dimensional hypercube (line) is $1$; the furthest in a 2-dimensional hypercube (square) is $\sqrt 2$ (opposite corners); the furthest in a 3-d hypercube is $\sqrt 3$ and so on. In general, the furthest two points in a $D$-dimensional hypercube will be $\sqrt D$.

You can actually compute these values analytically.  Write $\Uni_D$ for
the uniform distribution in $D$ dimensions.  The quantity we are
interested in computing is:
\begin{equation} \label{eq:knn:dist}
  \textit{avgDist}(D)
  = \Ep_{\vec a \sim \Uni_D} \Big[
    \Ep_{\vec b \sim \Uni_D} \Big[
      \norm{\vec a - \vec b}
      \Big] \Big]
\end{equation}
We can actually compute this in closed form and arrive
at $\textit{avgDist}(D) = \sqrt D / 3$. Because we know that the maximum distance between two points grows like $\sqrt D$, this says that the ratio between average distance and maximum distance converges to $1/3$.

What is more interesting, however, is the \emph{variance} of the distribution of distances. You can show that in $D$ dimensions, the variance is \emph{constant} $1/\sqrt{18}$, \emph{independent of $D$}. This means that when you look at (variance) divided-by (max distance), the variance behaves like $1/\sqrt{18 D}$, which means that the effective variance continues to shrink as $D$ grows\mycite{brin95nn}.

\Figure{knn:uniformhist}{histogram of distances in D=2,8,32,128,512}

When I first saw and re-proved this result, I was skeptical, as I
imagine you are.  So I implemented it.  In
Figure~\ref{fig:knn:uniformhist} you can see the results.  This
presents a \emph{histogram} of distances between random points in $D$
dimensions for $D \in \{1,2,3,10,20,100\}$.  As you can see, all of
these distances begin to concentrate around $0.4\sqrt{D}$, even for ``medium
dimension'' problems.

You should now be terrified: the only bit of information that KNN gets
is distances.  And you've just seen that in moderately high dimensions,
all distances becomes equal.  So then isn't it the case that KNN
simply cannot work?

\TODOFigure{knn:mnist}{histogram of distances in multiple D for mnist}

The answer has to be no.  The reason is that the data that we get is
\emph{not} uniformly distributed over the unit hypercube.  We can see
this by looking at two real-world data sets.  The first is an image
data set of hand-written digits (zero through nine); see
Section~\ref{sec:data:mnist}.  Although this data is originally in
$256$ dimensions ($16$ pixels by $16$ pixels), we can artifically
reduce the dimensionality of this data.  In Figure~\ref{fig:knn:mnist}
you can see the histogram of average distances between points in this
data at a number of dimensions.

As you can see from these histograms, distances have \emph{not}
concentrated around a single value.  This is very good news: it means
that there is hope for learning algorithms to work!  Nevertheless, the
moral is that high dimensions are weird.


% \section{Extensions to KNN}

% There are several fundamental problems with KNN classifiers.  First,
% some neighbors might be ``better'' than others.  Second, test-time
% performance scales badly as your number of training examples
% increases.  Third, it treats each dimension independently.  We will
% not address the third issue, as it has not really been solved (though
% it makes a great thought question!).

% \Figure{knn:badneighbors}{data set with 5nn, test point closest to
%   two negatives, then to three far positives}

% Regarding neighborliness, consider Figure~\ref{fig:knn:badneighbors}.
% Using $K=5$ nearest neighbors, the test point would be classified as
% positive.  However, we might actually believe that it should be
% classified negative because the two negative neighbors are \emph{much}
% closer than the three positive neighbors.

% \Figure{knn:badneighbors2}{same as previous with $\ep$ ball}

% There are at least two ways of addressing this issue.  The first is
% the \concept{$\ep$-ball} solution.  Instead of connecting each data
% point to some fixed number ($K$) of nearest neighbors, we simply
% connect it to \emph{all} neighbors that fall within some ball of
% radius $\ep$.  Then, the majority class of all the points in the $\ep$
% ball wins.  In the case of a tie, you would have to either guess, or
% report the majority class.  Figure~\ref{fig:knn:badneighbors2} shows an
% $\ep$ ball around the test point that happens to yield the proper
% classification.

% When using $\ep$-ball nearest neighbors rather than KNN, the
% hyperparameter changes from $K$ to $\ep$.  You would need to set it in
% the same way as you would for KNN.

% \thinkaboutit{One issue with $\ep$-balls is that the $\ep$-ball for
%   some test point might be empty.  How would you handle this?}

% An alternative to the $\ep$-ball solution is to do \concept{weighted
%   nearest neighbors}.  The idea here is to still consider the
% $K$-nearest neighbors of a test point, but give them uneven votes.
% Closer points get more vote than further points.  When classifying a
% point $\hat\vx$, the usual strategy is to give a training point
% $\vx_n$ a vote that decays exponentially in the distance between
% $\hat\vx$ and $\vx_n$.  Mathematically, the vote that neighbor $n$
% gets is:
% \begin{equation} \label{eq:expdecay}
% \exp\left[ - \frac 1 2 \norm{\hat\vx - \vx_n}^2 \right]
% \end{equation}
% Thus, nearby points get a vote very close to $1$ and far away points
% get a vote very close to $0$.  The overall prediction is positive if
% the sum of votes from positive neighbors outweighs the sum of votes
% from negative neighbors.

% \thinkaboutit{Could you combine the $\ep$-ball idea with the weighted
%   voting idea?  Does it make sense, or does one idea seem to trump the
%   other?}

% The second issue with KNN is scaling.  To predict the label of a
% single test point, we need to find the $K$ nearest neighbors of that
% test point in the training data.  With a standard implementation, this
% will take $\cO(N D + K \log K)$ time\sidenote{The $ND$ term comes from
%   computing distances between the test point and all training points.
%   The $K\log K$ term comes from finding the $K$ smallest values in the
%   list of distances, using a median-finding algorithm.  Of course,
%   $ND$ almost always dominates $K\log K$ in practice.}.  For very
% large data sets, this is impractical.

% \TODOFigure{knn:collapse}{two figures of points collapsed to mean, one
%   with good results and one with dire results}

% A first attempt to speed up the computation is to represent each class
% by a \emph{representative.}  A natural choice for a representative
% would be the mean.  We would collapse all positive examples down to
% their mean, and all negative examples down to their mean.  We could
% then just run $1$-nearest neighbor and check whether a test point is
% closer to the mean of the positive points or the mean of the negative
% points.  Figure~\ref{fig:knn:collapse} shows an example in which this
% would probably work well, and an example in which this would probably
% work poorly.  The problem is that collapsing each class to its mean is
% too aggressive.

% \TODOFigure{knn:collapse2}{data from previous bad case collapsed into
%   L=2 cluster and test point classified based on means and $1$-nn}

% A less aggressive approach is to make use of the $K$-means algorithm
% for clustering.  You can cluster the positive examples into $L$
% clusters (we are using $L$ to avoid variable overloading!) and then
% cluster the negative examples into $L$ separate clusters.  This is
% shown in Figure~\ref{fig:knn:collapse2} with $L=2$.  Instead of storing the
% entire data set, you would only store the \emph{means} of the $L$
% positive clusters and the means of the $L$ negative clusters.  At test
% time, you would run the $K$-nearest neighbors algorithm against these
% \emph{means} rather than against the full training set.  This leads to
% a much faster runtime of just $\cO(L D + K \log K)$, which is probably
% dominated by $LD$.

% \thinkaboutit{Clustering of classes was introduced as a way of making
%   things faster.  Will it make things worse, or could it help?}

\section{Further Reading}

TODO further reading



\begin{comment}
Similarities versus distances
Relationship to databases
Underfitting/overfitting by k
Weighted voting
From two classes to M classes
Decision boundaries for knn versus dts
Computational complexity
Reducing each class to one point: means
K-means clustering
\end{comment}

%%% Local Variables:
%%% mode: latex
%%% TeX-master: "courseml"
%%% End:
