\chapter{Probabilistic Modeling} \label{sec:prob}

\chapterquote{The world is noisy and messy. You need to deal with the noise and uncertainty.}{Daphne~Koller}

\begin{learningobjectives}
\item Define the generative story for a naive Bayes classifier.
\item Derive relative frequency as the solution to a constrained
  optimization problem.
\item Compare and contrast generative, conditional and discriminative
  learning.
\item Explain when generative models are likely to fail.
\item Derive logistic loss with an $\ell_2$ regularizer from a
  probabilistic perspective.
%\item Explain the maximum entropy principle and relate it to logistic
%  regression.
\end{learningobjectives}

\dependencies{}

\newthought{Many of the models and algorithms} you have learned about
thus far are relatively \emph{disconnected}.  There is an alternative
view of machine learning that unites and generalizes much of what you
have already learned.  This is the \concept{probabilistic modeling}
framework, in which you will explicitly think of learning as a problem
of \concept{statistical inference}.

In this chapter, you will learn about two flavors of probabilistic
models: generative and conditional.  You will see that many of the
approaches (both supervised and unsupervised) we have seen already can
be cast as probabilistic models.  Through this new view, you will be
able to develop learning algorithms that have inductive biases closer
to what you, as a designer, believe.  Moreover, the two chapters that
follow will make heavy use of the probabilistic modeling approach to
open doors to other learning problems.

\section{Classification by Density Estimation}

Recall from Chapter~\ref{sec:formal} that if we had access to the underlying probability distribution $\cD$, then we could form a Bayes optimal classifier as:
%
\begin{equation} \label{eq:prob:bayesopt2}
  f\xth{BO}(\hat x) = \arg\max_{\hat y \in \cY} \cD(\hat x, \hat y)
\end{equation}
%
Unfortunately, no one gave you this distribution, but the optimality of this approach
suggests that good way to build a classifier is to try to
\emph{estimate} $\cD$.  In other words, you try to learn a
distribution $\hat \cD$, which you hope to very similar to $\cD$, and
then use this distribution for classification.  Just as in the
preceding chapters, you can try to form your estimate of $\cD$ based
on a finite training set.

The most direct way that you can attempt to construct such a
probability distribution is to select a \emph{family} of parametric
distributions.  For instance, a Gaussian (or Normal) distribution is
parametric: its parameters are its mean and covariance.  The job of
learning is then to infer which parameters are ``best'' as far as the
observed training data is concerned, as well as whatever inductive
bias you bring.  A key assumption that you will need to make is that
the training data you have access to is drawn \concept{independently}
from $\cD$.  In particular, as you draw examples $(x_1,y_1) \sim \cD$
then $(x_2,y_2) \sim \cD$ and so on, the $n$th draw $(x_n,y_n)$ is
drawn from $\cD$ and \emph{does not} otherwise depend on the previous
$n-1$ samples.  This assumption is usually false, but is also usually
sufficiently close to being true to be useful.  Together with the
assumption that all the training data is drawn from the same
distribution $\cD$ leads to the \concept{i.i.d. assumption} or
\concept{independently and identically distributed} assumption.  This
is a key assumption in almost all of machine learning.

\begin{mathreview}{Rules of probability}
  A probability distribution $p$ specifies the likelihood of an event $e$, where $p(e) \in [0,1]$.
  It's often convenient to think of events as ``configurations of the world'', so $p(e)$ says ``how likely is it that the world is in configuration $e$.''
  Often world configurations are built up of smaller pieces, for instance you might say ``$e$ = the configuration in which it is rainy, windy and cold.''
  Formally, we might write this as ``$e = \{ \text{Weather}=\text{rainy}, \text{Wind}=\text{windy}, \text{Temperature}=\text{cold} \}$'', where we've used a convention that \concept{random variable}s (like Temperature) are capitalized and their instantiations (like cold) are lower case.
  Considering this event, we want to evaluate $p(\text{Weather}=\text{rainy}, \text{Wind}=\text{windy}, \text{Temperature}=\text{cold})$, or more generally $p(A=a, B=b, C=c)$ for some random variables $A$, $B$ and $C$, and some instantiations of those random variables $a$, $b$ and $c$ respectively.
  ~\\~\\
  There are a few standard rules of probability that we will use regularly:
  ~\\~\\
  \concept{sum-to-one}: if you sum over all possible configurations of the world, $p$ sums to one: $\sum_e p(E=e) = 1$.\\
  \concept{marginalization}: you can sum out one random variable to remove it from the world: $\sum_a p(A=a, B=b) = p(B=b)$.\\
  \concept{chain rule}: if a world configuration consists of two or more random variables, you can evaluate the likelihood of the world one step at a time: $p(A=a, B=b) = p(A=a) p(B=b \| A=a)$. Events are unordered, so you can also get $p(A=a, B=b) = p(B=b) p(A=a \| B=b)$.\\
  \concept{Bayes' rule}: combining the two chain rule equalities and dividing, we can relate a conditional probability in one direction with that in the other direction: $p(A=a \| B=b) = p(A=a) p(B=b \| A=a) / p(B=b)$.
\end{mathreview}

% Our underlying assumption for the majority of this book is that
% learning problems are characterized by some unknown probability
% distribution $\cD$ over input/output pairs $(x,y) \in \cX \times \cY$.
% Suppose that someone \emph{told} you what $\cD$ was.  In particular,
% they gave you a Python function \FUN{computeD} that took two inputs,
% $x$ and $y$, and returned the probability of that $x,y$ pair under
% $\cD$.  If you had access to such a function, classification becomes
% simple.  We can define the \concept{Bayes optimal classifier} as the
% classifier that, for any test input $\hat x$, simply returns the $\hat
% y$ that maximizes \FUN{computeD}$(\hat x, \hat y)$, or, more formally:
% %
% \begin{equation} \label{eq:prob:bayesopt}
%   f\xth{BO}(\hat x) = \arg\max_{\hat y \in \cY} \cD(\hat x, \hat y)
% \end{equation}
% %
% This classifier is optimal in one specific sense: of \emph{all
%   possible} classifiers, it achieves the smallest zero/one error.

% \begin{theorem}[Bayes Optimal Classifier] \label{thm:prob:bayesopt}
%   The Bayes Optimal Classifier $f\xth{BO}$ achieves minimal zero/one
%   error of any deterministic classifier.
% \end{theorem}

% This theorem assumes that you are comparing against
% \emph{deterministic} classifiers.  You can actually prove a stronger
% result that $f\xth{BO}$ is optimal for randomized classifiers as well,
% but the proof is a bit messier.  However, the intuition is the same:
% for a given $x$, $f\xth{BO}$ chooses the label with highest
% probability, thus \emph{minimizing} the probability that it makes an
% error.

% \begin{myproof}{\ref{thm:prob:bayesopt}}
%   Consider some other classifier $g$ that claims to be better than
%   $f$.  Then, there must be some $x$ on which $g(x) \neq f(x)$.  Fix
%   such an $x$.  Now, the probability that $f$ makes an error on this
%   particular $x$ is $1 - \cD(x, f\xth{BO}(x))$ and the probability
%   that $g$ makes an error on this $x$ is $1 - \cD(x, g(x))$.  But
%   $f\xth{BO}$ was chosen in such a way to \emph{maximize} $\cD(x,
%   f\xth{BO}(x))$, so this \emph{must} be greater than $\cD(x, g(x))$.
%   Thus, the probability that $f$ errs on this particular $x$ is
%   smaller than the probability that $g$ errs on it.  This applies to
%   any $x$ for which $f(x) \neq g(x)$ and therefore $f$ achieves
%   smaller zero/one error than any $g$.
% \end{myproof}

% The \concept{Bayes error rate} (or \concept{Bayes optimal error rate})
% is the error rate of the Bayes optimal classifier.  It is the best
% error rate you can ever hope to achieve on this classification problem
% (under zero/one loss).

% The take-home message is that if someone gave you access to the data
% distribution, forming an \emph{optimal} classifier would be trivial.
% Unfortunately, no one gave you this distribution, but this analysis
% suggests that good way to build a classifier is to try to
% \emph{estimate} $\cD$.  In other words, you try to learn a
% distribution $\hat \cD$, which you hope to very similar to $\cD$, and
% then use this distribution for classification.  Just as in the
% preceding chapters, you can try to form your estimate of $\cD$ based
% on a finite training set.

% The most direct way that you can attempt to construct such a
% probability distribution is to select a \emph{family} of parametric
% distributions.  For instance, a Gaussian (or Normal) distribution is
% parametric: it's parameters are its mean and covariance.  The job of
% learning is then to infer which parameters are ``best'' as far as the
% observed training data is concerned, as well as whatever inductive
% bias you bring.  A key assumption that you will need to make is that
% the training data you have access to is drawn \concept{independently}
% from $\cD$.  In particular, as you draw examples $(x_1,y_1) \sim \cD$
% then $(x_2,y_2) \sim \cD$ and so on, the $n$th draw $(x_n,y_n)$ is
% drawn from $\cD$ and \emph{does not} otherwise depend on the previous
% $n-1$ samples.  This assumption is usually false, but is also usually
% sufficiently close to being true to be useful.  Together with the
% assumption that all the training data is drawn from the same
% distribution $\cD$ leads to the \concept{i.i.d. assumption} or
% \concept{independently and identically distributed} assumption.  This
% is a key assumption in almost all of machine learning.

\section{Statistical Estimation}

Suppose you need to model a coin that is possibly biased (you can
think of this as modeling the \emph{label} in a binary classification
problem), and that you observe data \texttt{HHTH} (where
\texttt{H} means a flip came up heads and \texttt{T} means it came up
tails).  You can assume that all the flips came from the same coin,
and that each flip was independent (hence, the data was i.i.d.).
Further, you may choose to believe that the coin has a fixed
probability $\be$ of coming up heads (and hence $1-\be$ of coming up
tails).  Thus, the parameter of your model is simply the scalar $\be$.

\thinkaboutit{Describe a case in which at least one of the assumptions
  we are making about the coin flip is false.}

The most basic computation you might perform is \concept{maximum
  likelihood estimation}: namely, select the parameter $\be$ the
maximizes the probability of the data under that parameter.  In order
to do so, you need to compute the probability of the data:
%
\begin{align} \label{eq:prob:binomial}
  p_\be(D)
  &= p_\be(\texttt{HHTH})
     \becauseof{definition of $D$} \\
  &= p_\be(\texttt{H})
     p_\be(\texttt{H})
     p_\be(\texttt{T})
     p_\be(\texttt{H})
      \becauseof{data is independent} \\
  &= \be \be (1-\be) \be \\
  &= \be^3 (1-\be) \\
  &= \be^3 - \be^4
\end{align}
%
Thus, if you want the parameter $\be$ that maximizes the probability
of the data, you can take the derivative of $\be^3-\be^4$ with respect
to $\be$, set it equal to zero and solve for $\be$:
%
\begin{align}
  \frac {\partial} {\partial \be} \left[ \be^3 - \be^4 \right] &= 3 \be^2 -  4\be^3\\
       & 4\be^3 = 3\be^2  \\
  \Longleftrightarrow & 4\be = 3 \\
  \Longleftrightarrow & \be  = \frac 3 4
\end{align}
%
Thus, the maximum likelihood $\be$ is $0.75$, which is probably what
you would have selected by intuition.
%
You can solve this problem more generally as follows.  If you have
$H$-many heads and $T$-many tails, the probability of your data
sequence is $\be^H (1-\be)^T$.  You can try to take the derivative of
this with respect to $\be$ and follow the same recipe, but all of the
products make things difficult.  A more friendly solution is to work
with the \concept{log likelihood} or \concept{log probability}
instead.  The log likelihood of this data sequence is $H \log \be + T
\log (1-\be)$.  Differentiating with respect to $\be$, you get $H/\be
- T/(1-\be)$.  To solve, you obtain $H/\be = T/(1-\be)$ so $H (1-\be)
= T \be$.  Thus $H - H\be = T\be$ and so $H = (H+T) \be$, finally
yeilding that $\be = H/(H+T)$ or, simply, the fraction of observed
data that came up heads.  In this case, the maximum likelihood
estimate is nothing but the relative frequency of observing heads!

\thinkaboutit{How do you know that the solution of $\be=H/(H+T)$ is
  actually a maximum?}

Now, suppose that instead of flipping a coin, you're rolling a
$K$-sided die (for instance, to pick the label for a multiclass
classification problem).  You might model this by saying that there
are parameters $\th_1, \th_2, \dots, \th_K$ specifying, respectively,
the probabilities that any given side comes up on a role.  Since these
are themselves probabilities, each $\th_k$ should be at least zero,
and the sum of the $\th_k$s should be one.  Given a data set that
consists of $x_1$ rolls of $1$, $x_2$ rolls of $2$ and so on, the
probability of this data is $\prod_k \th_k^{x_k}$, yielding a log
probability of $\sum_k x_k \log \th_k$.  If you pick some particular
parameter, say $\th_3$, the derivative of this with respect to $\th_3$
is $x_3 / \th_3$, which you want to equate to zero.  This leads
to\dots $\th_3 \rightarrow \infty$.

This is obviously ``wrong.''  From the mathematical formulation, it's
correct: in fact, setting all of the $\th_k$s to $\infty$ \emph{does}
maximize $\prod_k \th_k^{x_k}$ for any (non-negative) $x_k$s.  The
problem is that you need to constrain the $\th$s to sum to one.  In
particular, you have a constraint that $\sum_k \th_k = 1$ that you
forgot to enforce.  A convenient way to enforce such constraints is
through the technique of \concept{Lagrange multipliers}.  To make this
problem consistent with standard minimization problems, it is
convenient to minimize negative log probabilities, instead of
maximizing log probabilities.  Thus, the \emph{constrainted}
optimization problem is:
%
\optimize{prob:multinomial}{\vec\th}{ \textcolor{darkblue}{- \sum_k x_k \log \th_k}}%
{\textcolor{darkergreen}{\sum_k \th_k - 1 = 0}}
%
The Lagrange multiplier approach involves adding a new variable $\la$
to the problem (called the \concept{Lagrange multiplier}) corresponding
to the constraint, and to use that to move the constraint into the
objective.  The result, in this case, is:
%
\optimizeLagrange{prob:multinomial:lagrange}{\la}{\vec\th}{%
  \textcolor{darkblue}{- \sum_k x_k \log \th_k} - \la \textcolor{darkergreen}{\left( \sum_k \th_k - 1 \right)}
}
%
Turning a constrained optimization problem into it's corresponding
\concept{Lagrangian} is straightforward.  The mystical aspect is why
it works.  In this case, the idea is as follows.  Think of $\la$ as an
adversary: $\la$ is trying to maximize this function (you're trying to
minimize it).  If you pick some parameters $\vec\th$ that actually
satisfy the constraint, then the green term in
Eq~\eqref{opt:prob:multinomial:lagrange} goes to zero, and therefore
$\la$ does not matter: the adversary cannot do anything.  On the other
hand, if the constraint is even \emph{slightly} unsatisfied, then
$\la$ can tend toward $+\infty$ or $-\infty$ to blow up the objective.
So, in order to have a non-infinite objective value, the optimizer
\emph{must} find values of $\vec\th$ that satisfy the constraint.

If we solve the \emph{inner} optimization of
Eq~\eqref{opt:prob:multinomial:lagrange} by differentiating with
respect to $\th_1$, we get $x_1 / \th_1 = \la$, yielding $\th_1 = x_1
/ \la$.  In general, the solution is $\th_k = x_k / \la$.  Remembering
that the goal of $\la$ is to enforce the sums-to-one constraint, we
can set $\la = \sum_k x_k$ and verify that this is a solution.  Thus,
our optimal $\th_k = x_k / \sum_k x_k$, which again completely
corresponds to intuition.

\section{Naive Bayes Models}

Now, consider the binary classification problem.  You are looking for
a \emph{parameterized} probability distribution that can describe the
training data you have.  To be concrete, your task might be to predict
whether a movie review is positive or negative (label) based on what
words (features) appear in that review.  Thus, the probability for a
\emph{single} data point can be written as:
%
\begin{equation} \label{eq:prob:nb0}
  \pth( (y, \vec x) )
  = \pth( y, x_1, x_2, \dots, x_D )
\end{equation}
%
The challenge in working with a probability distribution like
Eq~\eqref{eq:prob:nb0} is that it's a distribution over a \emph{lot}
of variables.  You can try to simplify it by applying the
\concept{chain rule} of probabilities:
%
\begin{align}
  \pth( x_1, x_2, \dots, x_D, y )
 &= \pth(y) \pth(x_1 \| y)
    \pth(x_2 \| y, x_1)
    \pth(x_3 \| y, x_1, x_2) \nonumber\\
 &\quad\quad   \cdots
    \pth(x_D \| y, x_1, x_2, \dots, x_{D-1})
  \\
&= \pth(y) \prod_d \pth(x_d \| y, x_1, \dots, x_{d-1})
\label{eq:prob:nbchain}
\end{align}
%
At this point, this equality is \emph{exact} for any probability
distribution.  However, it might be difficult to craft a probability
distribution for the $10000$th feature, given the previous $9999$.
Even if you could, it might be difficult to accurately estimate it.
At this point, you can make assumptions.  A classic assumption, called
the \concept{naive Bayes assumption}, is that \emph{the features are
  independent, conditioned on the label.}  In the movie review
example, this is saying that \emph{once you know that it's a positive
  review}, the probability that the word ``excellent'' appears is
independent of whether ``amazing'' also appeared.  (Note that this
does \emph{not} imply that these words are independent when you don't
know the label---they most certainly are not.)  Formally this
assumption states that:
%
\begin{equation} \label{eq:prob:nbassumption}
\textbf{Assumption:}\quad
  p(x_d \| y, x_{d'}) = p(x_d \| y) \quad,\quad \forall d \neq d'
\end{equation}
%
Under this assumption, you can simplify Eq~\eqref{eq:prob:nbchain} to:
%
\begin{align} \label{eq:prob:nb}
  \pth( (y, \vec x) )
&= \pth(y) \prod_d \pth(x_d \| y)
\becauseof{naive Bayes assumption}
\end{align}
%
At this point, you can start parameterizing $p$.  Suppose, for now,
that your labels are binary \emph{and} your features are also binary.
In this case, you could model the label as a biased coin, with
probability of heads (e.g., positive review) given by $\th_0$.  Then,
for each label, you can imagine having one (biased) coin for each
feature.  So if there are $D$-many features, you'll have $1+2D$ total
coins: one for the label (call it $\th_0$) and one for each
label/feature combination (call these $\vec\th_{+1}$ and as
$\vec\th_{-1}$).  In the movie review example, we might expect $\th_0
\approx 0.4$ (forty percent of movie reviews are positive) and also
that $\th_{+1}$ might give high probability to words like
``excellent'' and ``amazing'' and ``good'' and $\vec\th_{-1}$ might
give high probability to words like ``terrible'' and ``boring'' and
``hate''.  You can rewrite the probability of a single example as
follows, eventually leading to the log probability of the entire data
set:
%
\begin{align} \label{eq:prob:nbmult}
  \pth( (y, \vec x) )
&= \textcolor{darkergreen}{\pth(y)} \prod_d \textcolor{darkblue}{\pth(x_d \| y)}
\becauseof{naive Bayes assumption}\\
&= \left( \textcolor{darkergreen}{
      \th_0^{[y=+1]} (1-\th_0)^{[y=-1]}
    }\right)
  \prod_d
   \textcolor{darkblue}{
       \th_{(y),d}^{[x_d=1]}
       (1-\th_{(y),d})^{[x_d=0]}
       } \becauseof{model assumptions}
\end{align}

% \log \pth( (y, \vec x) )
% &= \textcolor{darkergreen}{[y=+1] \log \th_0
%    + [y=-1] \log (1-\th_0)}
% \becauseof{take logs}
% \nonumber\\
% &\quad\quad +
%    \sum_d
%    \left( \textcolor{darkblue}{ [x_d=1] \log \th_{(y),d}
%        + [x_d=0] \log (1-\th_{(y),d})} \right)
% \\
% \log\pth(D)
% &= \sum_n \Big[
% \textcolor{darkergreen}{[y_n=+1] \log \th_0
%    + [y_n=-1] \log (1-\th_0)}
% \becauseof{i.i.d. assumption}
% \nonumber\\
% &\quad\quad +
%    \sum_d
%    \left( \textcolor{darkblue}{ [x_{n,d}=1] \log \th_{(y_n),d}
%        + [x_{n,d}=0] \log (1-\th_{(y_n),d})} \right)
% \Big]
% \end{align}
%
Solving for $\th_0$ is identical to solving for the biased coin case
from before: it is just the relative frequency of positive labels in
your data (because $\th_0$ doesn't depend on $\vec x$ at all).  For
the other parameters, you can repeat the same exercise as before for
each of the $2D$ coins independently.  This yields:
%
\begin{align}
  \hat\th_0 &= \frac 1 N \sum_n [y_n = +1] \\
  \hat\th_{(+1),d} &= \frac {\sum_n [y_n = +1 \land x_{n,d} = 1]}  {\sum_n [y_n = +1]}\\
  \hat\th_{(-1),d} &= \frac {\sum_n [y_n = -1 \land x_{n,d} = 1]}  {\sum_n [y_n = -1]}
\end{align}
%
In the case that the features are \emph{not} binary, you need to
choose a different model for $p(x_d \| y)$.  The model we chose here
is the \concept{Bernouilli distribution}, which is effectively a
distribution over independent coin flips.  For other types of data,
other distributions become more appropriate.  The die example from
before corresponds to a \concept{discrete distribution}.  If the data
is continuous, you might choose to use a \concept{Gaussian
  distribution} (aka \concept{Normal distribution}).  The choice of
distribution is a form of \concept{inductive bias} by which you can
inject your knowledge of the problem into the learning algorithm.

\begin{mathreview}{Common probability distributions}
  There are a few common probability distributions that we use in this book.
  The first is the Bernouilli distribution, which models binary outcomes (like coin flips).
  A Bernouilli distribution, $\Ber(\th)$ is parameterized by a single scalar value $\th \in [0,1]$ that represents the probability of heads. The likelihood function is $\Ber(x \| \th) = \th^x (1-\th)^{1-x}$.
  The generalization of the Bernouilli to more than two possible outcomes (like rolls of a die) is the Discrete distribution, $\Disc(\vec \th)$. If the die has $K$ sides, then $\vec \th \in \R^K$ with all entries non-negative and $\sum_k \th_k = 1$. $\th_k$ is the probabability that the die comes up on side $k$. The likelihood function is $\Disc(x \| \vec\th) = \prod_k \th_k^{\Ind[x=k]}$.
  The Binomial distribution is just like the Bernouilli distribution but for multiple flips of the coin rather than a single flip; its likelihood is $\Bin(k \| n, \th) = n \choose k \th^k (1-\th)^{n-k}$, where $n$ is the number of flips and $k$ is the number of heads.
  The Multinomial distribution extends the Discrete distribution also to multiple rolls; its likelihood is $\Mult(\vec x \| n, \vec\th) = \frac {n!} {\prod_k x_k!} \prod_k \th_k^{x_k}$, where $n$ is the total number of rolls and $x_k$ is the number of times the die came up on side $k$ (so $\sum_k x_k = n$).
  The preceding distributions are all discrete.
  ~\\~\\
  There are two common continuous distributions we need. The first is the Uniform distribution, $\Uni(a,b)$ which is uniform over the closed range $[a,b]$. Its density function is $\Uni(x \| a,b) = \frac 1 {b-a} \Ind[x \in [a,b]]$.
  Finally, the Gaussian distribution is parameterized by a mean $\mu$ and variance $\si^2$ and has density $\Nor(x \| \mu,\si^2) = (2 \pi \si^2)^{-\frac 1 2} \exp\left[ - \frac 1 {2 \si^2} (x-\mu)^2\right]$.
\end{mathreview}


\section{Prediction}

Consider the \emph{predictions} made by the naive Bayes model with
Bernoulli features in Eq~\eqref{eq:prob:nbmult}.  You can better
understand this model by considering its decision boundary.  In the
case of probabilistic models, the decision boundary is the set of
inputs for which the likelihood of $y=+1$ is precisely $50\%$.  Or, in
other words, the set of inputs $\vx$ for which $p(y=+1 \| \vx) /
p(y=-1 \| \vx) = 1$.  In order to do this, the first thing to notice
is that $p(y \| \vx) = p(y,\vx) / p(\vx)$.  In the ratio, the $p(\vx)$
terms cancel, leaving $p(y=+1,\vx) / p(y=-1,\vx)$.  Instead of
computing this ratio, it is easier to compute the
\concept{log-likelihood ratio} (or LLR), $\log p(y=+1,\vx) - \log
p(y=-1,\vx)$, computed below:
%
\begin{align}
\text{LLR}
&= \log \left[ \th_0 \prod_d \th_{(+1),d}^{[x_d=1]} (1-\th_{(+1),d})^{[x_d=0]} \right] \nonumber\\
&\qquad - \log \left[ (1-\th_0) \prod_d \th_{(-1),d}^{[x_d=1]} (1-\th_{(-1),d})^{[x_d=0]} \right]
   \becauseof{model assumptions} \\
&= \log \th_0 - \log (1-\th_0)  + \sum_d [x_d=1] \left( \log \th_{(+1),d} - \log \th_{(-1),d} \right) \nonumber\\
&\qquad+ \sum_d [x_d=0] \left( \log (1-\th_{(+1),d}) - \log (1-\th_{(-1),d}) \right)
   \becauseof{take logs and rearrange} \\
&= \sum_d x_d \log \frac {\th_{(+1),d}} {\th_{(-1),d}}
 + \sum_d (1-x_d) \log \frac {1-\th_{(+1),d}} {1-\th_{(-1),d}}
 + \log \frac {\th_0} {1-\th_0}
    \becauseof{simplify log terms} \\
&= \sum_d x_d \left[ \log \frac {\th_{(+1),d}} {\th_{(-1),d}} - \log \frac {1-\th_{(+1),d}} {1-\th_{(-1),d}} \right]
 + \sum_d \log \frac {1-\th_{(+1),d}} {1-\th_{(-1),d}}
 + \log \frac {\th_0} {1-\th_0}
    \becauseof{group $x$-terms} \\
&= \dotp{\vx}{\vw} + b \\
w_d &= \log \frac {\th_{(+1),d} (1-\th_{(-1),d})} {\th_{(-1),d} (1-\th_{(+1),d})}
\quad,\quad
b   = \sum_d \log \frac {1-\th_{(+1),d}} {1-\th_{(-1),d}}
 + \log \frac {\th_0} {1-\th_0}
\end{align}
%
The result of the algebra is that the naive Bayes model has precisely
the form of a linear model!  Thus, like perceptron and many of the
other models you've previous studied, the decision boundary is linear.

% One advantage to working with probabilistic models is that the
% \emph{same} model can be used to predict under a variety of loss
% functions.  The prediction of ``choose the most likely class label''
% is optimal under zero/one loss, but alternative predictions can be
% made for other losses.  For example, consider the $\al$-Weighted
% Binary Classification problem from Chapter~\ref{sec:complex}.  In this
% problem, errors on positive examples cost $\al$, while errors on
% negative examples cost $1$.  Given an estimated probabilistic model
% $p(y \| \vx)$, you can minimize the \emph{expected error} by changing
% the threshold for prediction from $0.5$ to something else.  Suppose
% that the new threshold is $\be$; namely, if $p(y=+1\|\vx) > \be$ your
% function $f(\vx)$ predicts $+1$ and otherwise it predicts $-1$.  In
% this case, your error on \emph{positive examples} is $\al$ if
% $p(y=+1\|\vx) \leq \be$; your error on \emph{negative examples} is $1$
% if $p(y=+1\|\vx) > \be$ or equivalent $p(y=-1\|\vx) < 1-\be$.
% %
% \begin{align}
%    \Ep_{(\vx,y) \sim p} \left[ \al^{[y=+1]} [ p(y\|\vx)

% f(\vx) \neq y \right]
% &= \Ep_{(\vx,y) \sim p} \left[ \al^{[y=+1]} \frac {p(y=+1,\vx)} {p(y=-1,\vx)}


\section{Generative Stories}

A useful way to develop probabilistic models is to tell a
\concept{generative story}.  This is a \emph{fictional} story that
explains how you believe your training data came into existence.  To
make things interesting, consider a multiclass classification problem,
with continuous features modeled by independent Gaussians.  Since the
label can take values $1 \dots K$, you can use a discrete
distribution (die roll) to model it (as opposed to the Bernoulli
distribution from before):
%
\begin{enumerate}
  \item \textcolor{darkpurple}{For each example} $n = 1 \dots N$:
    \begin{enumerate}
      \item \textcolor{darkblue}{Choose a label} $y_n \sim \Disc(\vec\th)$
      \item \textcolor{darkred}{For each feature} $d = 1 \dots D$:
        \begin{enumerate}
          \item \textcolor{darkergreen}{Choose feature value} $x_{n,d} \sim \Nor(\mu_{y_n,d}, \si^2_{y_n,d})$
         \end{enumerate}
     \end{enumerate}
\end{enumerate}
%
This generative story can be directly translated into a likelihood
function by replacing the ``for each''s with products:
%
\begin{align}
  p(D)
  &=  \overbrace{
      \prod_n
       \underbrace{\th_{y_n}}_{\textsf{\textcolor{darkblue}{choose label}}}
       \underbrace{
       \prod_d
         \underbrace{\frac 1 {\sqrt{2 \pi \si_{y_n,d}^2}}
         \exp\left[
           -\frac 1 {2 \si_{y_n,d}^2} (x_{n,d} - \mu_{y_n,d})^2
           \right]}_{\textsf{\textcolor{darkergreen}{choose feature value}}}
         }_{\textsf{\textcolor{darkred}{for each feature}}}}
       ^{\textsf{\textcolor{darkpurple}{for each example}}}
\end{align}
%
You can take logs to arrive at the log-likelihood:
%
\begin{align}
\log p(D)
&=
\sum_n
  \left[
    \log \th_{y_n} +
    \sum_d
      -\frac 1 2 \log(\si_{y_n,d}^2)
      -\frac 1 {2 \si_{y_n,d}^2} (x_{n,d} - \mu_{y_n,d})^2
      \right]
  + \textit{const}
\end{align}
%
To optimize for $\vec\th$, you need to add a ``sums to one''
constraint as before.  This leads to the previous solution where the
$\th_k$s are proportional to the number of examples with label $k$.
In the case of the $\mu$s you can take a derivative with respect to,
say $\mu_{k,i}$ and obtain:
%
\begin{align}
   \frac {\partial \log p(D)} {\partial \mu_{k,i}}
&= \frac {\partial} {\partial \mu_{k,i}}
     - \sum_n \sum_d \frac 1 {2 \si_{y_n,d}^2} (x_{n,d} - \mu_{y_n,d})^2
   \becauseof{ignore irrelevant terms} \\
&= \frac {\partial} {\partial \mu_{k,i}}
     - \sum_{n:y_n=k} \frac 1 {2 \si_{k,i}^2} (x_{n,i} - \mu_{k,i})^2
   \becauseof{ignore irrelevant terms} \\
&= \sum_{n:y_n=k} \frac 1 {\si_{k,i}^2} (x_{n,i} - \mu_{k,i})
   \becauseof{take derivative}
\end{align}
%
Setting this equal to zero and solving yields:
%
\begin{align}
  \mu_{k,i} &= \frac {\sum_{n:y_n=k} x_{n,i}} {\sum_{n:y_n=k} 1}
\end{align}
%
Namely, the sample mean of the $i$th feature of the data points that
fall in class $k$.  A similar analysis for $\si_{k,i}^2$ yields:
%
\begin{align}
   \frac {\partial \log p(D)} {\partial \si^2_{k,i}}
&= \frac {\partial} {\partial \si^2_{k,i}}
     - \sum_{y:y_n=k} \left[ \frac 1 2 \log(\si_{k,i}^2) +
                            \frac 1 {2 \si_{k,i}^2} (x_{n,i} - \mu_{k,i})^2\right]
   \becauseof{ignore irrelevant terms} \\
&= - \sum_{y:y_n=k} \left[
       \frac 1 {2 \si_{k,i}^2}
     - \frac 1 {2 (\si_{k,i}^2)^2} (x_{n,i} - \mu_{k,i})^2 \right]
   \becauseof{take derivative} \\
&= \frac 1 {2 \si_{k,i}^4} \sum_{y:y_n=k} \left[ (x_{n,i} - \mu{k,i})^2 - \si_{k,i}^2 \right]
   \becauseof{simplify}
\end{align}
%
You can now set this equal to zero and solve, yielding:
%
\begin{align}
  \si_{k,i}^2 &=
    \frac {\sum_{n:y_n=k} (x_{n,i}-\mu_{k,i})^2} {\sum_{n:y_n=k} 1}
\end{align}
%
Which is just the sample variance of feature $i$ for class $k$.

\thinkaboutit{What would the estimate be if you decided that, for a
  given class $k$, all features had equal variance?  What if you
  assumed feature $i$ had equal variance for each class?  Under what
  circumstances might it be a good idea to make such assumptions?}

\section{Conditional Models}

In the foregoing examples, the task was formulated as attempting to
model the \concept{joint} distribution of $(\vx,y)$ pairs.  This may
seem wasteful: at prediction time, all you care about is $p(y\|\vx)$,
so why not model it directly?

Starting with the case of regression is actually somewhat simpler than
starting with classification in this case.  Suppose you ``believe''
that the relationship between the real value $y$ and the vector $\vx$
should be linear.  That is, you expect that $y = \dotp{\vw}{\vx}+b$
\emph{should hold} for some parameters $(\vw,b)$.  Of course, the data
that you get does not exactly obey this: that's fine, you can think of
deviations from $y = \dotp{\vw}{\vx}+b$ as \emph{noise}.  To form a
probabilistic model, you must assume some distribution over noise; a
convenient choice is zero-mean Gaussian noise.  This leads to the
following generative story:
%
\begin{enumerate}
  \item For each example $n=1 \dots N$:
    \begin{enumerate}
      \item Compute $t_n = \dotp{\vw}{\vx_n} + b$
      \item Choose noise $e_n \sim \Nor(0, \si^2)$
      \item Return $y_n = t_n + e_n$
    \end{enumerate}
\end{enumerate}
%
In this story, the variable $t_n$ stands for ``target.''  It is the
noiseless variable that you do not get to observe.  Similarly $e_n$ is
the error (noise) on example $n$.  The value that you actually get to
observe is $y_n = t_n + e_n$.  See Figure~\ref{fig:prob:regression}.

\Figure{prob:regression}{pictorial view of targets versus labels}

A basic property of the Gaussian distribution is additivity.  Namely,
that if $a \sim \Nor(\mu,\si^2)$ and $b = a + c$, then $b \sim
\Nor(\mu+c,\si^2)$.  Given this, from the generative story above, you
can derive a shorter generative story:
%
\begin{enumerate}
  \item For each example $n=1 \dots N$:
    \begin{enumerate}
      \item Choose $y_n \sim \Nor( \dotp{\vw}{\vx_n} + b, \si^2)$
    \end{enumerate}
\end{enumerate}
%
Reading off the log likelihood of a dataset from this generative
story, you obtain:
%
\begin{align}
   \log p(D)
&= \sum_n \left[
     - \frac 1 2 \log(\si^2)
     - \frac 1 {2 \si^2} \left( \dotp{\vw}{\vx_n} + b - y_n \right)^2
     \right]
   \becauseof{model assumptions}\\
&= - \frac 1 {2 \si^2}  \sum_n \left( \dotp{\vw}{\vx_n} + b - y_n \right)^2
   + \textit{const}
   \becauseof{remove constants}
\end{align}
%
This is precisely the linear regression model you encountered in
Section~\ref{sec:loss:reg}!  To minimize the \emph{negative} log
probability, you need only solve for the regression coefficients
$\vw,b$ as before.

In the case of binary classification, using a Gaussian noise model
does not make sense.  Switching to a Bernoulli model, which describes
binary outcomes, makes more sense.  The only remaining difficulty is
that the parameter of a Bernoulli is a value between zero and one (the
probability of ``heads'') so your model must produce such values.  A
classic approach is to produce a real-valued target, as before, and
then transform this target into a value between zero and one, so that
$-\infty$ maps to $0$ and $+\infty$ maps to $1$.  A function that does
this is the logistic function\footnote{Also called the
  \concept{sigmoid function} because of it's ``S''-shape.}, defined
below and plotted in Figure~\ref{fig:prob:sigmoid}:
%
\Figure{prob:sigmoid}{sketch of logistic function}
%
\begin{equation}
\textsf{Logistic function:} \quad
\si(z) = \frac 1 {1 + \exp[-z]} = \frac {\exp z}{1 + \exp z}
\end{equation}
%
The logistic function has several nice properties that you can verify
for yourself: $\si(-z) = 1 - \si(z)$ and $\partial \si/\partial z = z
\si^2(z)$.

Using the logistic function, you can write down a generative story for
binary classification:

\begin{enumerate}
  \item For each example $n=1 \dots N$:
    \begin{enumerate}
      \item Compute $t_n = \si\left(\dotp{\vw}{\vx_n} + b\right)$
      \item Compute $z_n \sim \Ber(t_n)$
      \item Return $y_n = 2 z_n - 1$ (to make it $\pm 1$)
    \end{enumerate}
\end{enumerate}

The log-likelihood for this model is:
%
\begin{align}
   \log p(D)
&= \sum_n \Big[ [y_n=+1] \log \si\left(\dotp{\vw}{\vx_n} + b\right)\nonumber\\
&\qquad
               + [y_n=-1] \log \si\left(-\dotp{\vw}{\vx_n} - b\right)
            \Big]
   \becauseof{model and properties of $\si$}\\
&= \sum_n \log \si\left( y_n \left(\dotp{\vw}{\vx_n} + b\right)\right)
   \becauseof{join terms} \\
&= - \sum_n \log \left[ 1 + \exp\left( - y_n \left(\dotp{\vw}{\vx_n} +
      b\right)\right) \right]
    \becauseof{definition of $\si$}\\
&= - \sum_n \ell\xth{log}(y_n, \dotp{\vw}{\vx_n} + b)
    \becauseof{definition of $\ell\xth{log}$}
\end{align}

As you can see, the log-likelihood is \emph{precisely} the negative of
(a scaled version of) the logistic loss from Chapter~\ref{sec:loss}.
This model is the \concept{logistic regression} model, and this is
where logistic loss originally derived from.

%One key question is: when should one prefer conditional models to
%joint model

TODO: conditional versus joint

\section{Regularization via Priors}

In the foregoing discussion, parameters of the model were selected
according to the maximum likelihood criteria: find the parameters
$\vec\th$ that maximize $p_{\vec\th}(D)$.  The trouble with this
approach is easy to see even in a simple coin flipping example.  If
you flip a coin twice and it comes up heads both times, the maximum
likelihood estimate for the bias of the coin is $100\%$: it will
\emph{always} come up heads.  This is true even if you had only
flipped it once!  If course if you had flipped it one million times
and it had come up heads every time, \emph{then} you might find this
to be a reasonable solution.

This is clearly undesirable behavior, especially since data is
expensive in a machine learning setting.  One solution (there are
others!) is to seek parameters that balance a tradeoff between the
likelihood of the data and some prior belief you have about what
values of those parameters are likely.  Taking the case of the
logistic regression, you might a priori believe that small values of
$\vw$ are more likely than large values, and choose to represent this
as a Gaussian prior on each component of $\vw$.

The \concept{maximum a posteriori} principle is a method for
incorporating both data and prior beliefs to obtain a more balanced
parameter estimate.  In abstract terms, consider a probabilistic model
over data $D$ that is parameterized by parameters $\th$.  If you think
of the parameters as just another random variable, then you can write
this model as $p(D \| \th)$, and maximum likelihood amounts to
choosing $\th$ to maximize $p(D \| \th)$.  However, you might instead
wish to maximize the probability of the \emph{parameters}, given the
data.  Namely, maximize $p(\th \| D)$.  This term is known as the
\concept{posterior} distribution on $\th$, and can be computed by
Bayes' rule:
%
\begin{align}
  \underbrace{p(\th \| D)}_{\text{posterior}}
  &= \frac {
       \overbrace{p(\th)}^{\text{prior}}
       \overbrace{p(D \| \th)}^{\text{likelihood}}
     } {
       \underbrace{p(D)}_{\text{evidence}}
     }
 \quad\text{, where}\quad
 p(D) = \int \ud\th p(\th) p(D \| \th)
\end{align}
%
This reads: the \concept{posterior} is equal to the \concept{prior}
times the \concept{likelihood} divided by the
\concept{evidence}.\footnote{The evidence is sometimes called the
  \concept{marginal likelihood}.}  The evidence is a scary-looking
term (it has an integral!) but note that from the perspective of
seeking parameters $\th$ than maximize the posterior, the evidence is
just a constant (it does not depend on $\th$) and therefore can be
ignored.

Returning to the logistic regression example with Gaussian priors on
the weights, the \concept{log posterior} looks like:
%
\begin{align}
   \log p(\th \| D)
&= - \sum_n \ell\xth{log}(y_n, \dotp{\vw}{\vx_n} + b)
   - \sum_d \frac 1 {2 \si^2} w_d^2
   + \textit{const}
   \becauseof{model definition}\\
&= - \sum_n \ell\xth{log}(y_n, \dotp{\vw}{\vx_n} + b)
   - \frac 1 {2 \si^2} \norm{\vw}^2
\end{align}
%
and therefore reduces to a regularized logistic function, with a
squared $2$-norm regularizer on the weights.  (A $1$-norm regularizer
can be obtained by using a Laplace prior on $\vw$ rather than a
Gaussian prior on $\vw$.)

%\section{Maximum Entropy Principle}
%calculus of variations
%derivation of logistic regression

\section{Further Reading}

TODO


\begin{comment}
   - The i.i.d. assumption, and extended chain rule
   - Logistic regression as a conditional model
   - Maximum entropy models
   - Generative modeling, non-naive Bayes and memorizing data
   - Naive Bayes and feature independence
   - Naive Bayes as a linear model
\end{comment}

%%% Local Variables:
%%% mode: latex
%%% TeX-master: "courseml"
%%% End:
