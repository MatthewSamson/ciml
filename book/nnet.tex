\chapter{Neural Networks} \label{sec:nnet}

\chapterquote{TODO}{}

\begin{learningobjectives}
\item Explain the biological inspiration for multi-layer neural
  networks.
\item Construct a two-layer network that can solve the XOR problem.
\item Implement the back-propogation algorithm for training
  multi-layer networks.
\item Explain the trade-off between depth and breadth in network
  structure.
\item Contrast neural networks with radial basis functions with
  $k$-nearest neighbor learning.
\end{learningobjectives}

\dependencies{}

\newthought{The first learning models} you learned about (decision
trees and nearest neighbor models) created complex,
\concept{non-linear} decision boundaries.  We moved from there to the
perceptron, perhaps the most classic linear model.  At this point, we
will move \emph{back} to non-linear learning models, but using all
that we have learned about linear learning thus far.

This chapter presents an extension of perceptron learning to
non-linear decision boundaries, taking the biological inspiration of
neurons even further.  In the perceptron, we thought of the input data
point (e.g., an image) as being directly connected to an output (e.g.,
label).  This is often called a \concept{single-layer network} because
there is one layer of weights.  Now, instead of directly connecting
the inputs to the outputs, we will insert a layer of ``hidden'' nodes,
moving from a single-layer network to a \concept{multi-layer network}.
But introducing a non-linearity at inner layers, this will give us
non-linear decision boundaires.  In fact, such networks are able to
express almost any function we want, not just linear functions.  The
trade-off for this flexibility is increased complexity in parameter
tuning and model design.

\section{Bio-inspired Multi-Layer Networks}

One of the major weaknesses of linear models, like perceptron and the
regularized linear models from the previous chapter, is that they are
linear!  Namely, they are unable to learn arbitrary decision
boundaries.  In contrast, decision trees and $K$NN \emph{could} learn
arbitrarily complicated decision boundaries.

\Figure{nnet:twolayer}{picture of a two-layer network with 5
  inputs and two hidden units}

One approach to doing this is to chain together a collection of
perceptrons to build more complex \concept{neural networks}.  An
example of a \concept{two-layer network} is shown in
Figure~\ref{fig:nnet:twolayer}.  Here, you can see five inputs
(features) that are fed into two \concept{hidden units}.  These hidden
units are then fed in to a single \concept{output unit}.  Each edge in
this figure corresponds to a different weight.  (Even though it looks
like there are three layers, this is called a two-layer network
because we don't count the inputs as a real layer.  That is, it's two
layers of \emph{trained} weights.)

Prediction with a neural network is a straightforward generalization
of prediction with a perceptron.  First you compute activations of the
nodes in the hidden unit based on the inputs and the input weights.
Then you compute activations of the output unit given the hidden unit
activations and the second layer of weights.

The only major difference between this computation and the perceptron
computation is that the hidden units compute a \emph{non-linear}
function of their inputs.  This is usually called the
\concept{activation function} or \concept{link function}.  More
formally, if $w_{i,d}$ is the weights on the edge connecting input $d$
to hidden unit $i$, then the activation of hidden unit $i$ is computed
as:
%
\begin{equation}
  h_i = f\left( \dotp{\vw_i}{\vx} \right)
\end{equation}
%
Where $f$ is the link function and $\vw_i$ refers to the
vector of weights feeding in to node $i$.

One example link function is the \concept{sign} function.  That is, if
the incoming signal is negative, the activation is $-1$.  Otherwise
the activation is $+1$.  This is a potentially useful activiation
function, but you might already have guessed the problem with it: it
is non-differentiable.

\Figure{nnet:tanh}{picture of sign versus tanh}

EXPLAIN BIAS!!!

A more popular link function is the \concept{hyperbolic tangent}
function, $\tanh$.  A comparison between the sign function and the
$\tanh$ function is in Figure~\ref{fig:nnet:tanh}.  As you can see, it
is a reasonable approximation to the sign function, but is convenient
in that it is differentiable.\sidenote{It's derivative is just
  $1-\tanh^2(x)$.}  Because it looks like an ``S'' and because the
Greek character for ``S'' is ``Sigma,'' such functions are usually
called \concept{sigmoid} functions.

\newalgorithm%
  {nnet:twolayerpredict}%
  {\FUN{TwoLayerNetworkPredict}(\VAR{$\mat W$}, \VAR{$\vec v$}, \VAR{$\hat\vx$})}
  {
\FOR{\VAR{i} = \CON{1} \TO \VAR{number of hidden units}}
\SETST{$h_i$}{$\tanh(\dotp{\VARm{\vec w_i}}{\VARm{\hat\vx}})$}
\COMMENT{compute activation of hidden unit $i$}
\ENDFOR
\RETURN $\dotp{\VARm{\vec v}}{\VARm{\vec h}}$
\COMMENT{compute output unit}
}


Assuming for now that we are using $\tanh$ as the link function, the
overall prediction made by a two-layer network can be computed using
Algorithm~\ref{alg:nnet:twolayerpredict}.  This function takes a
matrix of weights $\mat W$ corresponding to the first layer weights
and a vector of weights $\vec v$ corresponding to the second layer.
You can write this entire computation out in one line as:
%
\begin{align}
\hat y
&= \sum_i v_i \tanh( \dotp{\vec w_i}{\hat\vx} ) \\
&= \dotp{\vec v}{\tanh( \mat W \hat\vx )} \label{eq:nnet:matrixeq}
\end{align}
%
Where the second line is short hand assuming that $\tanh$ can take a
vector as input and product a vector as output.

\thinkaboutit{Is it necessary to use a link function at all?  What
  would happen if you just used the identify function as a link?}

\TableSize{large}{nnet:xor}{Small XOR data set.}{r|rrr}{
  $y$ & $x_0$ & $x_1$ & $x_2$ \\
  \hline
  +1 & +1 & +1 & +1 \\
  +1 & +1 & -1 & -1 \\
  -1 & +1 & +1 & -1 \\
  -1 & +1 & -1 & +1}

The claim is that two-layer neural networks are more expressive than
single layer networks (i.e., perceptrons).  To see this, you can
construct a very small two-layer network for solving the XOR problem.
For simplicity, suppose that the data set consists of four data
points, given in Table~\ref{tab:nnet:xor}.  The classification rule is
that $y=+1$ if and only if $x_1=x_2$, where the features are just $\pm
1$.

You can solve this problem using a two layer network with two hidden
units.  The key idea is to make the first hidden unit compute an
``or'' function: $x_1 \lor x_2$.  The second hidden unit can compute
an ``and'' function: $x_1 \land x_2$.  The the output can combine
these into a single prediction that mimics XOR.  Once you have the
first hidden unit activate for ``or'' and the second for ``and,'' you
need only set the output weights as $-2$ and $+1$, respectively.

\thinkaboutit{Verify that these output weights will actually give you XOR.}

To achieve the ``or'' behavior, you can start by setting the bias to
$-0.5$ and the weights for the two ``real'' features as both being
$1$.  You can check for yourself that this will do the ``right thing''
if the link function were the sign function.  Of course it's not, it's
$\tanh$.  To get $\tanh$ to mimic $\sign$, you need to make the dot
product either really really large or really really small.  You can
accomplish this by setting the bias to $-500,000$ and both of the two
weights to $1,000,000$.  Now, the activation of this unit will be just
slightly above $-1$ for $\vx = \langle -1,-1 \rangle$ and just
slightly below $+1$ for the other three examples.

\thinkaboutit{This shows how to create an ``or'' function.  How can
  you create an ``and'' function?}

At this point you've seen that one-layer networks (aka perceptrons)
can represent any linear function and only linear functions.  You've
also seen that two-layer networks can represent non-linear functions
like XOR.  A natural question is: do you get additional
representational power by moving beyond two layers?  The answer is
partially provided in the following Theorem, due originally to George
Cybenko for one particular type of link function, and extended later
by Kurt Hornik to arbitrary link functions.

\begin{theorem}[Two-Layer Networks are Universal Function Approximators] \label{thm:nnet:twolayer}
%
  Let $F$ be a continuous function on a bounded subset of
  $D$-dimensional space.  Then there exists a two-layer neural network
  $\hat F$ with a finite number of hidden units that approximate $F$
  arbitrarily well.  Namely, for all $\vx$ in the domain of $F$,
  $\ab{F(\vx) - \hat F(\vx)} < \ep$.
\end{theorem}

Or, in colloquial terms ``two-layer networks can approximate any
function.''

This is a remarkable theorem.  Practically, it says that if you give
me a function $F$ and some error tolerance parameter $\ep$, I can
construct a two layer network that computes $F$.  In a sense, it says
that going from one layer to two layers completely changes the
representational capacity of your model.

When working with two-layer networks, the key question is: how many
hidden units should I have?  If your data is $D$ dimensional and you
have $K$ hidden units, then the total number of parameters is
$(D+2)K$.  (The first $+1$ is from the bias, the second is from the
second layer of weights.)  Following on from the heuristic that you
should have one to two examples for each parameter you are trying to
estimate, this suggests a method for choosing the number of hidden
units as roughly $\lfloor \frac N D \rfloor$.  In other words, if you
have tons and tons of examples, you can safely have lots of hidden
units.  If you only have a few examples, you should probably restrict
the number of hidden units in your network.

The number of units is both a form of inductive bias and a form of
regularization.  In both view, the number of hidden units controls how
complex your function will be.  Lots of hidden units $\Rightarrow$
very complicated function.  %Figure~\ref{fig:nnet:units} shows training
%and test error for neural networks trained with different numbers of
%hidden units.
As the number increases, training performance continues
to get better.  But at some point, test performance gets worse because
the network has overfit the data.

\section{The Back-propagation Algorithm}

The back-propagation algorithm is a classic approach to training
neural networks.  Although it was not originally seen this way, based
on what you know from the last chapter, you can summarize
back-propagation as:
%
\begin{equation}
\text{back-propagation} = \text{gradient descent} + \text{chain rule}
\end{equation}
%
More specifically, the set up is \emph{exactly} the same as before.
You are going to optimize the weights in the network to minimize some
objective function.  The only difference is that the predictor is no
longer linear (i.e., $\hat y = \dotp{\vw}{\vx}+b$) but now non-linear
(i.e., $\dotp{\vec v}{\tanh( \mat W \hat\vx )}$).  The only question
is how to do gradient descent on this more complicated objective.

For now, we will ignore the idea of regularization.  This is for two
reasons.  The first is that you already know how to deal with
regularization, so everything you've learned before applies.  The
second is that \emph{historically}, neural networks have not been
regularized.  Instead, people have used \concept{early stopping} as a
method for controlling overfitting.  Presently, it's not obvious which
is a better solution: both are valid options.

To be completely explicit, we will focus on optimizing squared error.
Again, this is mostly for historic reasons.  You could easily replace
squared error with your loss function of choice.  Our overall
objective is:
%
\optimizeuc{nnet:twolayer}{\mat W,\vec v}{%
  \sum_n \frac 1 2 \left( \textcolor{darkblue}{y_n -
     \sum_i v_i f(\dotp{\vec w_i}{\vx_n})}
     \right)^2}
%
Here, $f$ is some link function like $\tanh$.

The easy case is to differentiate this with respect to $\vec v$: the
weights for the output unit.  Without even doing any math, you should
be able to guess what this looks like.  The way to think about it is
that from $\vec v$s perspective, it is just a linear model, attempting
to minimize squared error.  The only ``funny'' thing is that its
inputs are the activations $\vec h$ rather than the examples $\vx$.
So the gradient with respect to $\vec v$ is just as for the linear
case.

To make things notationally more convenient, let $e_n$ denote the
\emph{error} on the $n$th example (i.e., the blue term above), and let
$\vec h_n$ denote the vector of hidden unit activations on that
example.  Then:
%
\begin{equation}
\grad_{\vec v} = -\sum_n e_n \vec h_n
\end{equation}
%
This is exactly like the linear case.  One way of interpreting this
is: how would the output weights have to change to make the prediction
better?  This is an easy question to answer because they can easily
measure how their changes affect the output.

The more complicated aspect to deal with is the weights corresponding
to the \emph{first} layer.  The reason this is difficult is because
the weights in the first layer aren't necessarily trying to produce
specific values, say $0$ or $5$ or $-2.1$.  They are simply trying to
produce activations that get fed to the output layer.  So the change
they want to make depends crucially on how the output layer interprets
them.

Thankfully, the chain rule of calculus saves us.  Ignoring the sum
over data points, we can compute:
%
\begin{align}
\cL(\mat W) &=
\frac 1 2 \left( \textcolor{darkblue}{y -
     \sum_i v_i f(\dotp{\vec w_i}{\vx})}
     \right)^2
\\
\frac {\partial \cL} {\partial \vec w_i}
&= \frac {\partial \cL} {\partial f_i}
   \frac {\partial f_i}   {\partial \vec w_i}
\\
\frac {\partial \cL} {\partial f_i}
&= -\left(y - \sum_i v_i f(\dotp{\vec w_i}{\vx})\right) v_i
= -e v_i
\\
\frac {\partial f_i} {\partial \vec w_i}
&= f'(\dotp{\vec w_i}{\vx}) \vx
\end{align}
%
Putting this together, we get that the gradient with respect to
$\vw_i$ is:
%
\begin{equation}
\grad_{\vw_i}
= -e v_i f'(\dotp{\vw_i}{\vx}) \vec x
\end{equation}
%
Intuitively you can make sense of this.  If the overall error of the
predictor ($e$) is small, you want to make small steps.  If $v_i$ is
small for hidden unit $i$, then this means that the output is not
particularly sensitive to the activation of the $i$th hidden unit.
Thus, its gradient should be small.  If $v_i$ flips sign, the gradient
at $\vw_i$ should also flip signs.  The name
\concept{back-propagation} comes from the fact that you propagate
gradients backward through the network, starting at the end.

\newalgorithm%
  {nnet:twolayertrain}%
  {\FUN{TwoLayerNetworkTrain}(\VAR{$\mat D$}, \VAR{$\eta$}, \VAR{K}, \VAR{MaxIter})}
  {
\SETST{$\mat W$}{$D \times K$ matrix of small random values}
\COMMENT{initialize input layer weights}
\SETST{$\vec v$}{$K$-vector of small random values}
\COMMENT{initialize output layer weights}
\FOR{\VAR{iter} = \CON{1} \dots \VAR{MaxIter}}
\SETST{$\mat G$}{$D \times K$ matrix of zeros}
\COMMENT{initialize input layer gradient}
\SETST{$\vec g$}{$K$-vector of zeros}
\COMMENT{initialize output layer gradient}
\FORALL{(\VAR{$\vx$},\VAR{$y$}) $\in$ \VAR{$\mat D$}}
\FOR{\VAR{i} = \CON{1} \TO \VAR{K}}
\SETST{$a_i$}{$\dotp{\VARm{\vec w_i}}{\VARm{\hat\vx}}$}
\SETST{$h_i$}{$\tanh(\VARm{a_i})$}
\COMMENT{compute activation of hidden unit $i$}
\ENDFOR
\SETST{$\hat y$}{$\dotp{\VARm{\vec v}}{\VARm{\vec h}}$}
\COMMENT{compute output unit}
\SETST{$e$}{$\VARm{y} - \VARm{\hat y}$}
\COMMENT{compute error}
\SETST{$\vec g$}{$\VARm{\vec g} - \VARm{e} \VARm{\vec h}$}
\COMMENT{update gradient for output layer}
\FOR{\VAR{i} = \CON{1} \TO \VAR{K}}
\SETST{$\mat G_i$}{$\VARm{\mat G_i} - \VARm{e} \VARm{v_i} (\CON{1} - \tanh^2(\VARm{a_i})) \VARm{\vx}$}
\COMMENT{update gradient for input layer}
\ENDFOR
\ENDFOR
\SETST{$\mat W$}{$\VARm{\mat W} - \VARm{\eta} \VARm{\mat G}$}
  \COMMENT{update input layer weights}
\SETST{$\vec v$}{$\VARm{\vec v} - \VARm{\eta} \VARm{\vec g}$}
  \COMMENT{update output layer weights}
\ENDFOR
\RETURN \VAR{$\mat W$}, \VAR{$\vec v$}
}

The complete instantiation of gradient descent for a two layer network
with $K$ hidden units is sketched in
Algorithm~\ref{alg:nnet:twolayertrain}.  Note that this really is
\emph{exactly} a gradient descent algorithm; the only different is
that the computation of the gradients of the input layer is moderately
complicated.

\thinkaboutit{What would happen to this algorithm if you wanted to
  optimize exponential loss instead of squared error?  What if you
  wanted to add in weight regularization?}

As a bit of practical advice, implementing the back-propagation
algorithm can be a bit tricky.  Sign errors often abound.  A useful
trick is first to keep $\mat W$ fixed and work on just training $\vec
v$.  Then keep $\vec v$ fixed and work on training $\mat W$.  Then put
them together.

\thinkaboutit{If you like matrix calculus, derive the same algorithm
  starting from Eq~\eqref{eq:nnet:matrixeq}.}

\section{Initialization and Convergence of Neural Networks}

Based on what you know about linear models, you might be tempted to
initialize all the weights in a neural network to zero.  You might
also have noticed that in Algorithm~\ref{alg:nnet:twolayertrain}, this
is not what's done: they're initialized to small random values.  The
question is why?

The answer is because an initialization of $\mat W = \mat 0$ and $\vec
v = \vec 0$ will lead to ``uninteresting'' solutions.  In other words,
if you initialize the model in this way, it will eventually get stuck
in a bad local optimum.  To see this, first realize that on any
example $\vx$, the activation $h_i$ of the hidden units will all be
zero since $\mat W = \mat 0$.  This means that on the first iteration,
the gradient on the output weights ($\vec v$) will be zero, so they
will stay put.  Furthermore, the gradient $w_{1,d}$ for the $d$th
feature on the $i$th unit will be \emph{exactly} the same as the
gradient $w_{2,d}$ for the same feature on the second unit.  This
means that the weight matrix, after a gradient step, will change in
\emph{exactly the same way} for every hidden unit.  Thinking through
this example for iterations $2\dots$, the values of the hidden units
will \emph{always} be exactly the same, which means that the weights
feeding in to any of the hidden units will be exactly the same.
Eventually the model will converge, but it will converge to a solution
that does not take advantage of having access to the hidden units.

This shows that neural networks are \emph{sensitive} to their
initialization.  In particular, the function that they optimize is
\concept{non-convex}, meaning that it might have plentiful local
optima.  (One of which is the trivial local optimum described in the
preceding paragraph.)  In a sense, neural networks \emph{must} have
local optima.  Suppose you have a two layer network with two hidden
units that's been optimized.  You have weights $\vw_1$ from inputs to
the first hidden unit, weights $\vw_2$ from inputs to the second
hidden unit and weights $(v_1,v_2)$ from the hidden units to the
output.  If I give you back another network with $\vw_1$ and $\vw_2$
swapped, and $v_1$ and $v_2$ swapped, the network computes
\emph{exactly} the same thing, but with a markedly different weight
structure.  This phenomena is known as \concept{symmetric modes}
(``mode'' referring to an optima) meaning that there are symmetries in
the weight space.  It would be one thing if there were lots of modes
and they were all symmetric: then finding one of them would be as good
as finding any other.  Unfortunately there are additional local
optima that are \emph{not} global optima.

\Figure{nnet:converge}{convergence of randomly initialized networks}

Random initialization of the weights of a network is a way to address
\emph{both} of these problems.  By initializing a network with small
random weights (say, uniform between $-0.1$ and $0.1$), the network is
unlikely to fall into the trivial, symmetric local optimum.  Moreover,
by training a collection of networks, each with a different random
initialization, you can often obtain better solutions that with just
one initialization.  In other words, you can train ten networks with
different random seeds, and then pick the one that does best on
held-out data.  Figure~\ref{fig:nnet:converge} shows prototypical
\emph{test-set} performance for ten networks with different random
initialization, plus an eleventh plot for the trivial symmetric
network initialized with zeros.

One of the typical complaints about neural networks is that they are
finicky.  In particular, they have a rather large number of knobs to
tune:
%
\begin{enumerate}
\item The number of layers
\item The number of hidden units per layer
\item The gradient descent learning rate $\eta$
\item The initialization
\item The stopping iteration or weight regularization
\end{enumerate}
%
The last of these is minor (early stopping is an easy regularization
method that does not require much effort to tune), but the others are
somewhat significant.  Even for two layer networks, having to choose
the number of hidden units, and then get the learning rate and
initialization ``right'' can take a bit of work.  Clearly it can be
automated, but nonetheless it takes time.

Another difficulty of neural networks is that their weights can be
difficult to interpret.  You've seen that, for linear networks, you
can often interpret high weights as indicative of positive examples
and low weights as indicative of negative examples.  In multilayer
networks, it becomes very difficult to try to understand what the
different hidden units are doing.

%TODO: maybe have something about doing images?


\section{Beyond Two Layers}
\MoveNextFigure{-10cm}
\Figure{nnet:layered}{multi-layer network}

The definition of neural networks and the back-propagation algorithm
can be generalized beyond two layers to any arbitrary directed acyclic
graph.  In practice, it is most common to use a layered network like
that shown in Figure~\ref{fig:nnet:layered} unless one has a very
strong reason (aka inductive bias) to do something different.
However, the view as a directed graph sheds a different sort of
insight on the back-propagation algorithm.

\MoveNextFigure{-1cm}
\Figure{nnet:dag}{DAG network}

Suppose that your network structure is stored in some directed acyclic
graph, like that in Figure~\ref{fig:nnet:dag}.  We index nodes in this
graph as $u,v$.  The activation \emph{before} applying non-linearity
at a node is $a_u$ and after non-linearity is $h_u$.  The graph has a
single sink, which is the output node $y$ with activation $a_y$ (no
non-linearity is performed on the output unit).  The graph has
$D$-many inputs (i.e., nodes with no parent), whose activations $h_u$
are given by an input example.  An edge $(u,v)$ is from a parent to a
child (i.e., from an input to a hidden unit, or from a hidden unit to
the sink).  Each edge has a weight $w_{u,v}$.  We say that
$\textit{par}(u)$ is the set of parents of $u$.

There are two relevant algorithms: forward-propagation and
back-propagation.  Forward-propagation tells you how to compute the
activation of the sink $y$ given the inputs.  Back-propagation
computes derivatives of the edge weights for a given input.

\newalgorithm%
  {nnet:forwardprop}%
  {\FUN{ForwardPropagation}{($\vx$)}}
  {
\FORALL{input nodes $u$}
\SETST{$h_u$}{corresponding feature of $\vx$}
\ENDFOR
\FORALL{nodes $v$ in the network whose parent's are computed}
\SETST{$a_v$}{$\sum_{u \in \textit{par}(v)} w_{(u,v)} h_u$}
\SETST{$h_v$}{$\tanh(a_v)$}
\ENDFOR
\RETURN $a_y$
}
\newalgorithm%
  {nnet:backprop}%
  {\FUN{BackPropagation}{($\vx$, $y$)}}
  {
\STATE{run \FUN{ForwardPropagation}(\VAR{$\vx$}) to compute activations}
\SETST{$e_y$}{$y - a_y$}
  \COMMENT{compute overall network error}
\FORALL{nodes $v$ in the network whose error $e_v$ is computed}
\FORALL{$u \in \textit{par}(v)$}
\SETST{$g_{u,v}$}{$- e_v h_u$}
  \COMMENT{compute gradient of this edge}
\SETST{$e_u$}{$e_u + e_v w_{u,v} (1 - \tanh^2(a_u))$}
  \COMMENT{compute the ``error'' of the parent node}
\ENDFOR
\ENDFOR
\RETURN all gradients $g_e$
}

\MoveNextFigure{-8cm}
\Figure{nnet:forwardprop}{picture of forward prop}

The key aspect of the \concept{forward-propagation} algorithm is to
iteratively compute activations, going deeper and deeper in the DAG.
Once the activations of all the parents of a node $u$ have been
computed, you can compute the activation of node $u$.  This is spelled
out in Algorithm~\ref{alg:nnet:forwardprop}.  This is also explained
pictorially in Figure~\ref{fig:nnet:forwardprop}.

\Figure{nnet:backprop}{picture of back prop}

Back-propagation (see Algorithm~\ref{alg:nnet:backprop}) does the
opposite: it computes gradients top-down in the network.  The key idea
is to compute an \emph{error} for each node in the network.  The error
at the output unit is the ``true error.''  For any input unit, the
error is the amount of gradient that we see coming from our children
(i.e., higher in the network).  These errors are computed backwards in
the network (hence the name \concept{back-propagation}) along with the
gradients themselves.  This is also explained pictorially in
Figure~\ref{fig:nnet:backprop}.

Given the back-propagation algorithm, you can directly run gradient
descent, using it as a subroutine for computing the gradients.

\section{Breadth versus Depth}

At this point, you've seen how to train two-layer networks and how to
train arbitrary networks.  You've also seen a theorem that says that
two-layer networks are universal function approximators.  This begs
the question: if two-layer networks are so great, why do we care about
deeper networks?

To understand the answer, we can borrow some ideas from CS theory,
namely the idea of \concept{circuit complexity}.  The goal is to show
that there are functions for which it might be a ``good idea'' to use
a deep network.  In other words, there are functions that will require
a huge number of hidden units if you force the network to be shallow,
but can be done in a small number of units if you allow it to be
deep.  The example that we'll use is the \concept{parity function}
which, ironically enough, is just a generalization of the XOR
problem.  The function is defined over binary inputs as:
%
\begin{align}
\textit{parity}(\vx)
&= \sum_d \vx_d \mod 2 \\
&= \brack{ 1 & \text{if the number of $1$s in $\vx$ is odd} \\
           0 & \text{if the number of $1$s in $\vx$ is even} }
\end{align}
%
\TODOFigure{nnet:paritydeep}{deep function for computing parity}
%
It is easy to define a circuit of depth $\cO(\log_2 D)$ with
$\cO(D)$-many gates for computing the parity function.  Each gate is
an XOR, arranged in a complete binary tree, as shown in
Figure~\ref{fig:nnet:paritydeep}.  (If you want to disallow XOR as a
gate, you can fix this by allowing the depth to be doubled and
replacing each XOR with an AND, OR and NOT combination, like you did
at the beginning of this chapter.)

This shows that if you are allowed to be deep, you can construct a
circuit with that computes parity using a number of hidden units that
is linear in the dimensionality.  So can you do the same with shallow
circuits?  The answer is no.  It's a famous result of circuit
complexity that parity requires exponentially many gates to compute in
constant depth.  The formal theorem is below:

\begin{theorem}[Parity Function Complexity] \label{thm:nnet:parity}
  Any circuit of depth $K < \log_2 D$ that computes the parity
  function of $D$ input bits must contain $\cO{e^D}$ gates.
\end{theorem}

This is a very famous result because it shows that constant-depth
circuits are less powerful that deep circuits.  Although a neural
network isn't exactly the same as a circuit, the is generally believed
that the same result holds for neural networks.  At the very least,
this gives a strong indication that depth might be an important
consideration in neural networks.

\thinkaboutit{What is it about neural networks that makes it so that
  the theorem about circuits does not apply directly?}

One way of thinking about the issue of breadth versus depth has to do
with the number of \emph{parameters} that need to be estimated.  By
the heuristic that you need roughly one or two examples for every
parameter, a deep model could potentially require exponentially fewer
examples to train than a shallow model!

This now flips the question: if deep is potentially so much better,
why doesn't everyone use deep networks?  There are at least two
answers.  First, it makes the \concept{architecture selection} problem
more significant.  Namely, when you use a two-layer network, the only
hyperparameter to choose is how many hidden units should go in the
middle layer.  When you choose a deep network, you need to choose how
many layers, and what is the width of all those layers.  This can be
somewhat daunting.

A second issue has to do with training deep models with
back-propagation.  In general, as back-propagation works its way down
through the model, the sizes of the gradients shrink.  You can work
this out mathematically, but the intuition is simpler.  If you are the
beginning of a very deep network, changing one single weight is
unlikely to have a significant effect on the output, since it has to
go through so many other units before getting there.  This directly
implies that the derivatives are small.  This, in turn, means that
back-propagation essentially never moves far from its initialization
when run on very deep networks.

\thinkaboutit{While these small derivatives might make training
  difficult, they might be \emph{good} for other reasons: what
  reasons?}

Finding good ways to train deep networks is an active research area.
There are two general strategies.  The first is to attempt to
initialize the weights better, often by a \concept{layer-wise}
initialization strategy.  This can be often done using unlabeled data.
After this initialization, back-propagation can be run to tweak the
weights for whatever classification problem you care about.  A second
approach is to use a more complex optimization procedure, rather than
gradient descent.  You will learn about some such procedures later in
this book.

\section{Basis Functions}

At this point, we've seen that: (a) neural networks can mimic linear
functions and (b) they can learn more complex functions.  A reasonable
question is whether they can mimic a $K$NN classifier, and whether
they can do it efficiently (i.e., with not-too-many hidden units).

A natural way to train a neural network to mimic a $K$NN classifier is
to replace the sigmoid link function with a \concept{radial basis
  function} (RBF).  In a \concept{sigmoid network} (i.e., a network
with sigmoid links), the hidden units were computed as $h_i =
\tanh(\dotp{\vw_i,\vx})$.  In an \concept{RBF network}, the hidden
units are computed as:
%
\begin{equation}
h_i = \exp\left[ - \ga_i \norm{\vw_i - \vx}^2 \right]
\end{equation}
%
\TODOFigure{nnet:rbfpicture}{a one-D picture of RBF bumps}

\TODOFigure{nnet:unitsymbols}{picture of nnet with sigmoid/rbf units}
%
In other words, the hidden units behave like little Gaussian ``bumps''
centered around locations specified by the vectors $\vw_i$.  A
one-dimensional example is shown in Figure~\ref{fig:nnet:rbfpicture}.
The parameter $\ga_i$ specifies the \emph{width} of the Gaussian bump.
If $\ga_i$ is large, then only data points that are really close to
$\vw_i$ have non-zero activations.  To distinguish sigmoid networks
from RBF networks, the hidden units are typically drawn with sigmoids
or with Gaussian bumps, as in Figure~\ref{fig:nnet:unitsymbols}.

Training RBF networks involves finding good values for the Gaussian
widths, $\ga_i$, the centers of the Gaussian bumps, $\vw_i$ and the
connections between the Gaussian bumps and the output unit, $\vec v$.
This can all be done using back-propagation.  The gradient terms for
$\vec v$ remain unchanged from before, the the derivates for the other
variables differ (see Exercise~\ref{}).

One of the big questions with RBF networks is: where should the
Gaussian bumps be centered?  One can, of course, apply
back-propagation to attempt to find the centers.  Another option is to
specify them ahead of time.  For instance, one potential approach is
to have one RBF unit per data point, centered on that data point.  If
you carefully choose the $\ga$s and $\vec v$s, you can obtain
something that looks nearly identical to distance-weighted $K$NN by
doing so.  This has the added advantage that you can go futher, and
use back-propagation to \emph{learn} good Gaussian widths ($\ga$) and
``voting'' factors ($\vec v$) for the nearest neighbor algorithm.

\thinkaboutit{Consider an RBF network with one hidden unit per
  training point, centered at that point.  What bad thing might happen
  if you use back-propagation to estimate the $\ga$s and $\vec v$ on
  this data if you're not careful?  How could you be careful?}

% If you use \emph{fewer} hidden units than data point, you can think
% of this technique as a \concept{locally weighted linear model}.
% This is easiest to picture in the \concept{regression} setting.
% Figure~\ref{fig:nnet:lwlr} shows an example of using locally
% weighted linear regression to fit a very non-linear curve using a
% relatively small number of bumps.  As with all neural networks,
% choosing the right number of hidden units (aka the
% \concept{architecture selection} problem) is one of the challenges
% in RBF network and locally weighted linear models.

\section{Further Reading}

TODO further reading


\begin{comment}
   - Multilayer perceptron
   - From linear to non-linear
   - Architecture choice
   - Basis functions
   - Each training point as a basic function
\end{comment}


%%% Local Variables:
%%% mode: latex
%%% TeX-master: "courseml"
%%% End:
